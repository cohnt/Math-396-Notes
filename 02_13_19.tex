\documentclass[10pt,letterpaper]{article}
\usepackage[utf8]{inputenc}
\usepackage[intlimits]{amsmath}
\usepackage{amsfonts}
\usepackage{amssymb}
\usepackage{ragged2e}
\usepackage[letterpaper, margin=1in]{geometry}
\usepackage{graphicx}
\usepackage{cancel}
\usepackage{mathtools}
\usepackage{tabularx}
\usepackage{arydshln}
\usepackage{tensor}
\usepackage{array}
\usepackage{xcolor}
\usepackage[boxed]{algorithm}
\usepackage[noend]{algpseudocode}
\usepackage{listings}
\usepackage{textcomp}
\usepackage[pdf,tmpdir,singlefile]{graphviz}
\usepackage{mathrsfs}
\usepackage{bbm}
\usepackage{tikz}
\usepackage{enumitem}
\usepackage{arydshln}
\usepackage{relsize}
\usepackage{multicol}

%%%%%%%%%%%%%%%%%%%%%%%%%%%%%
% Formatting commands
%%%%%%%%%%%%%%%%%%%%%%%%%%%%%
\newcommand{\n}{\hfill\break}
\newcommand{\up}{\vspace{-\baselineskip}}
\newcommand{\lemma}[1]{\par\noindent\settowidth{\hangindent}{\textbf{Lemma: }}\textbf{Lemma: }#1}
\newcommand{\defn}[1]{\par\noindent\settowidth{\hangindent}{\textbf{Defn: }}\textbf{Defn: }#1\n}
\newcommand{\thm}[1]{\par\noindent\settowidth{\hangindent}{\textbf{Thm: }}\textbf{Thm: }#1\n}
\newcommand{\prop}[1]{\par\noindent\settowidth{\hangindent}{\textbf{Prop: }}\textbf{Prop: }#1\n}
\newcommand{\cor}[1]{\par\noindent\settowidth{\hangindent}{\textbf{Cor: }}\textbf{Cor: }#1\n}
\newcommand{\ex}[1]{\par\noindent\settowidth{\hangindent}{\textbf{Ex: }}\textbf{Ex: }#1\n}
\newcommand{\exer}[1]{\par\noindent\settowidth{\hangindent}{\textbf{Exer: }}\textbf{Exer: }#1\n}
\newcommand{\proven}{\;$\square$\n}
\newcommand{\problem}[1]{\par\noindent{#1}\n}
\newcommand{\problempart}[2]{\par\noindent\indent{}\settowidth{\hangindent}{\textbf{(#1)} \indent{}}\textbf{(#1)} #2\n}
\newcommand{\ptxt}[1]{\textrm{\textnormal{#1}}}
\newcommand{\inlineeq}[1]{\centerline{$\displaystyle #1$}}
\newcommand{\pageline}{\noindent\rule{\textwidth}{0.1pt}}

%%%%%%%%%%%%%%%%%%%%%%%%%%%%%
% Math commands
%%%%%%%%%%%%%%%%%%%%%%%%%%%%%
% Set Theory
\newcommand{\card}[1]{\left|#1\right|}
\newcommand{\set}[1]{\left\{#1\right\}}
\newcommand{\setmid}{\;\middle|\;}
\newcommand{\ps}[1]{\mathcal{P}\left(#1\right)}
\newcommand{\pfinite}[1]{\mathcal{P}^{\ptxt{finite}}\left(#1\right)}
\newcommand{\naturals}{\mathbb{N}}
\newcommand{\N}{\naturals}
\newcommand{\integers}{\mathbb{Z}}
\newcommand{\Z}{\integers}
\newcommand{\rationals}{\mathbb{Q}}
\newcommand{\Q}{\rationals}
\newcommand{\reals}{\mathbb{R}}
\newcommand{\R}{\reals}
\newcommand{\complex}{\mathbb{C}}
\newcommand{\C}{\complex}
\newcommand{\halfPlane}{\mathbb{H}}
\let\H\relax
\newcommand{\H}{\halfPlane}
\newcommand{\comp}{^{\complement}}
\DeclareMathOperator{\Hom}{Hom}
\newcommand{\Ind}{\mathbbm{1}}
\newcommand{\cut}{\setminus}
\DeclareMathOperator{\elem}{elem}

% Graph Theory
\let\deg\relax
\DeclareMathOperator{\deg}{deg}
\newcommand{\degp}{\ptxt{deg}^{+}}
\newcommand{\degn}{\ptxt{deg}^{-}}
\newcommand{\precdot}{\mathrel{\ooalign{$\prec$\cr\hidewidth\hbox{$\cdot\mkern0.5mu$}\cr}}}
\newcommand{\succdot}{\mathrel{\ooalign{$\cdot\mkern0.5mu$\cr\hidewidth\hbox{$\succ$}\cr\phantom{$\succ$}}}}
\DeclareMathOperator{\cl}{cl}
\DeclareMathOperator{\affdim}{affdim}

% Probability
\newcommand{\Prob}{\mathbb{P}}
\newcommand{\Avg}{\mathbb{E}}
\DeclareMathOperator{\Var}{Var}
\DeclareMathOperator{\cov}{cov}

% Standard Math
\newcommand{\inv}{^{-1}}
\newcommand{\abs}[1]{\left|#1\right|}
\newcommand{\ceil}[1]{\left\lceil{}#1\right\rceil{}}
\newcommand{\floor}[1]{\left\lfloor{}#1\right\rfloor{}}
\newcommand{\conj}[1]{\overline{#1}}
\newcommand{\of}{\circ}
\newcommand{\tri}{\triangle}
\newcommand{\inj}{\hookrightarrow}
\newcommand{\surj}{\twoheadrightarrow}
\newcommand{\ndiv}{\nmid}
\renewcommand{\epsilon}{\varepsilon}
\newcommand{\divides}{\mid}
\newcommand{\ndivides}{\nmid}
\DeclareMathOperator{\lcm}{lcm}
\DeclareMathOperator{\sgn}{sgn}
\newcommand{\map}[4]{\!\!\!\begin{array}[t]{rcl}#1 & \!\!\!\!\to & \!\!\!\!#2\\ #3 & \!\!\!\!\mapsto & \!\!\!\!#4\end{array}}
\newcommand{\bigsum}[2]{\smashoperator[lr]{\sum_{\scalebox{#1}{$#2$}}}}

% Linear Algebra
\newcommand{\Id}{\textrm{\textnormal{Id}}}
\newcommand{\im}{\textrm{\textnormal{im}}}
\newcommand{\norm}[1]{\abs{\abs{#1}}}
\newcommand{\tpose}{^{T}}
\newcommand{\iprod}[1]{\left<#1\right>}
\DeclareMathOperator{\trace}{tr}
\newcommand{\chgBasMat}[3]{\!\!\tensor*[_{#1}]{\left[#2\right]}{_{#3}}}
\newcommand{\vecBas}[2]{\tensor*[]{\left[#1\right]}{_{#2}}}
\DeclareMathOperator{\GL}{GL}
\DeclareMathOperator{\Mat}{Mat}
\DeclareMathOperator{\vspan}{span}
\DeclareMathOperator{\rank}{rank}
\newcommand{\V}[1]{\vec{#1}}
\DeclareMathOperator{\proj}{proj}
\DeclareMathOperator{\compProj}{comp}
\DeclareMathOperator{\row}{row}

% Multilinear Algebra
\newcommand{\Lsym}{\L}
\let\L\relax
\DeclareMathOperator{\L}{\mathscr{L}}
\DeclareMathOperator{\A}{\mathcal{A}}
\DeclareMathOperator{\Alt}{Alt}
\DeclareMathOperator{\Sym}{Sym}
\newcommand{\ot}{\otimes}
\newcommand{\ox}{\otimes}
\DeclareMathOperator{\asc}{asc}
\DeclareMathOperator{\asSet}{set}
\DeclareMathOperator{\sort}{sort}
\DeclareMathOperator{\ringA}{\mathring{A}}

% Topology
\newcommand{\closure}[1]{\overline{#1}}
\newcommand{\uball}{\mathcal{U}}
\DeclareMathOperator{\Int}{Int}
\DeclareMathOperator{\Ext}{Ext}
\DeclareMathOperator{\Bd}{Bd}
\DeclareMathOperator{\rInt}{rInt}
\DeclareMathOperator{\ch}{ch}
\DeclareMathOperator{\ah}{ah}
\newcommand{\Tau}{\mathlarger{\mathlarger{\mathlarger{\mathlarger{\tau}}}}}

% Analysis
\DeclareMathOperator{\Graph}{Graph}
\DeclareMathOperator{\epi}{epi}
\DeclareMathOperator{\hypo}{hypo}
\DeclareMathOperator{\supp}{supp}
\newcommand{\lint}[2]{\underset{#1}{\overset{#2}{{\color{black}\underline{{\color{white}\overline{{\color{black}\int}}\color{black}}}}}}}
\newcommand{\uint}[2]{\underset{#1}{\overset{#2}{{\color{white}\underline{{\color{black}\overline{{\color{black}\int}}\color{black}}}}}}}
\newcommand{\alignint}[2]{\underset{#1}{\overset{#2}{{\color{white}\underline{{\color{white}\overline{{\color{black}\int}}\color{black}}}}}}}
\newcommand{\extint}{\ptxt{ext}\int}
\newcommand{\extalignint}[2]{\ptxt{ext}\alignint{#1}{#2}}
\newcommand{\conv}{\ast}

% Proofs
\newcommand{\st}{s.t.}
\newcommand{\unique}{!}
\newcommand{\iffdef}{\overset{\ptxt{def}}{\Leftrightarrow}}
\newcommand{\eqdef}{\overset{\ptxt{def}}{=}}
\newcommand{\eqVertical}{\rotatebox[origin=c]{90}{=}}
\newcommand{\mapsfrom}{\mathrel{\reflectbox{\ensuremath{\mapsto}}}}
\newcommand{\mapsdown}{\rotatebox[origin=c]{-90}{$\mapsto$}\mkern2mu}
\newcommand{\mapsup}{\rotatebox[origin=c]{90}{$\mapsto$}\mkern2mu}
\newcommand{\from}{\!\mathrel{\reflectbox{\ensuremath{\to}}}}

% Brackets
\newcommand{\paren}[1]{\left(#1\right)}
\renewcommand{\brack}[1]{\left[#1\right]}
\renewcommand{\brace}[1]{\left\{#1\right\}}
\newcommand{\ang}[1]{\left<#1\right>}

% Algorithms
\algrenewcommand{\algorithmiccomment}[1]{\hskip 1em \texttt{// #1}}
\algrenewcommand\algorithmicrequire{\textbf{Input:}}
\algrenewcommand\algorithmicensure{\textbf{Output:}}
\newcommand{\parSymbol}{\P}
\renewcommand{\P}{\ptxt{\textbf{P}}}
\newcommand{\NP}{\ptxt{\textbf{NP}}}
\newcommand{\NPC}{\ptxt{\textbf{NP-Complete}}}
\newcommand{\NPH}{\ptxt{\textbf{NP-Hard}}}
\newcommand{\EXP}{\ptxt{\textbf{EXP}}}

%%%%%%%%%%%%%%%%%%%%%%%%%%%%%
% Other commands
%%%%%%%%%%%%%%%%%%%%%%%%%%%%%
\newcommand{\flag}[1]{\textbf{\textcolor{red}{#1}}}
\newcommand{\uSym}{\u}
\let\u\relax
\newcommand{\u}[1]{\underline{#1}}
\newcommand{\bSym}{\b}
\let\b\relax
\newcommand{\b}[1]{\textbf{#1}}
\newcommand{\iSym}{\i}
\let\i\relax
\newcommand{\i}[1]{\textit{#1}}

%%%%%%%%%%%%%%%%%%%%%%%%%%%%%
% Make l's curvy in math environments
%%%%%%%%%%%%%%%%%%%%%%%%%%%%%
\mathcode`l="8000
\begingroup
\makeatletter
\lccode`\~=`\l
\DeclareMathSymbol{\lsb@l}{\mathalpha}{letters}{`l}
\lowercase{\gdef~{\ifnum\the\mathgroup=\m@ne \ell \else \lsb@l \fi}}%
\endgroup

\newcommand{\B}{
    \begin{tikzpicture}
    \filldraw [fill=red, draw=black] (0, 0) rectangle (0.37, 0.45);
    \draw [line width=0.5mm, white ] (0.1,0.08) -- (0.1,0.38);
    \draw[line width=0.5mm, white ] (0.1, 0.35) .. controls (0.2, 0.35) and (0.4, 0.2625) .. (0.1, 0.225);
    \draw[line width=0.5mm, white ] (0.1, 0.225) .. controls (0.2, 0.225) and (0.4, 0.1625) .. (0.1, 0.1);
    \end{tikzpicture}
}

\author{Thomas Cohn}
\title{More Orientation Special Cases\\Vector Calculus vs. Exterior Calculus}
\date{2/13/19} % Can also use \today

\begin{document}
\maketitle
\setlength\RaggedRightParindent{\parindent}
\RaggedRight

\par\noindent Orientation special caes of a $k$-manifold in $\R^{n}$:
\begin{enumerate}[label=(\arabic*)]
	\item $k=n$: See notes from 2/11/19.
	\item $k=n-1$: See notes from 2/11/19.
	\item $k=1$: Let $X$ be a $1$-manifold. Then an orientation on $X$ can be matched with a choice of continuous ``forward-pointing'' unit tangent. For orientation-preserving coordinate patch $\alpha$, and $\vec{p}\in{}X$, $\vec{q}=\alpha\inv(\vec{p})$, we have unit tangent $\vec{T}(\vec{p})=\frac{\alpha'(\vec{p})}{\norm{\alpha'(\vec{p})}}$.
	\item[(3a)] $X=\partial{}M^{2\ptxt{-mfd}}$. Then roughly speaking, when looking at the loop from the ``outside'' of $M$, the orientation on $X$ is counterclockwise.
	\item[(3b)] $M^{2\ptxt{-mfd}}\subseteq\R^{2}$, $X=\partial{}M$. Then for tangent and normal vectors $\vec{T}(\vec{p})$ and $\vec{N}(\vec{p})$, $\vec{T}(\vec{p})$ is just $\vec{N}(\vec{p})$ rotated $90^{\circ}$ counterclockwise.
	\item $k=0$: Recall that a compact $0$-manifold is a finite set. A compact, connected $0$-manifold is a singleton. Singletons have $2$ orientations (denoted $\pm{}1$).\n
	So an orientation on $X^{0\ptxt{-mfd}}$ is just a mapping $\epsilon:X\to\set{\pm{}1}$. For compact oriented $0$-manifold $X$ and $f$ $0$-form,\n
	\inlineeq{
		\int_{X}f\eqdef\sum_{\vec{x}\in{}X}\epsilon(\vec{x})f(\vec{x})
	}
	If $X=\partial{}M^{1\ptxt{-mfd}}$, then\n
	\inlineeq{
		\int_{M}df=\int_{\partial{}M}f=f(b)+(-1)f(a)=f(b)-f(a)
	}
\end{enumerate}

\par\noindent Building on HW5 \#1:\n
For $M$ oriented $k$-manifold, orientation-preserving $\map{\alpha:U}{V\subset{}M}{\vec{q}}{\vec{p}}$, $\omega$ $k$-form in a neighborhood of $M$
\[
\star\left\{\begin{array}{l}\alpha^{*}\omega=f(\vec{x})\wedge{}dx_{1}\wedge\cdots\wedge{}dx_{k}\\
\omega\;\left\{\begin{array}{l}
\ptxt{positive for $M$ at $\vec{p}$}\\
\ptxt{negative for $M$ at $\vec{p}$}\\
\ptxt{integral for $M$ at $\vec{p}$}
\end{array}\right\}\iffdef\left\{
\begin{array}{l}
f(\vec{q})>0\\
f(\vec{q})<0\\
f(\vec{q})=0
\end{array}
\right\}
\end{array}\right.
\]

\exer{$\omega$ integral at $\vec{p}$ $\Leftrightarrow$ $\omega(\vec{p})(\vec{v_{1}},\ldots,\vec{v_{k}})=0$ when $\vec{v_{i}}\in\Tau_{\vec{p}}M$.}

\par\noindent $M$ is an integral manifold for $\omega$ $\iffdef$ $\omega$ is integral for $M$ at all $\vec{p}\in{}M$.\n

\par\noindent Conversely, given $\omega$ nowhere integral on $M$, we get an orientation on $M$. Declare $\omega$ to be positive, call $\alpha$ orientation-preserving if $\star$ holds. We also get an orientation on each $\Tau_{\vec{p}}M$: each basis (or \u{frame}) $\vec{v_{1}},\ldots,\vec{v_{k}}$ for $\Tau_{\vec{p}}M$ is positively oriented $\Leftrightarrow$ $\omega(\vec{p})(\vec{v_{1}},\ldots,\vec{v_{k}})>0$.\n

\thm{36.2 For $\omega$ $k$-form on a neighborhoold of $M$, a compact oriented $k$-manifold,\n
$\displaystyle\int_{M}f=\int_{M}\lambda\,dV\;\ptxt{where }\map{\lambda:M}{\R}{\vec{p}}{\omega(\vec{p})(\vec{v_{1}},\ldots,\vec{v_{k}})}\ptxt{ for any positively-oriented orthonormal basis for $\Tau_{\vec{p}}M$}$}

\par\noindent Rceall that we did $k=1$ on November 28: $\displaystyle\in_{M}\omega=\int_{M}\omega\cdot\vec{T}\,ds=\int_{M}\omega(\vec{T})\,ds$.\n

\par\noindent We would like to be able to do extended integrals over manifolds. What is $\int_{M}\omega$ for $M$ some non-compact oriented manifold?\n

\defn{$\displaystyle\extint_{M}\omega\eqdef\extint_{M}\lambda\,dV=\underbrace{\extint_{M}\lambda_{+}\,dV}_{\mathclap{=\sup\set{\int_{N}\lambda_{+}\,dV\;:\;N^{\ptxt{cpt $k$-mfd}}\subseteq{}M}}}-\extint_{M}\lambda_{-}\,dV$}

\par\noindent (Note that we allow $N$ to inherit its orientation from $M$.)\n

\begin{center}
\begin{tabular}{l|l}
Exterior Calculus in $\R^{2}$ & Vector Calculus in $\R^{2}$\\ \hline
Diffeomorphisms & Isometries (translations, rotations, and reflections)\\
$0$-form $f$ & Scalar Function $f$\\
$1$-form $\omega=\alpha\,dx+\beta\,dy$ & Vector field $\vec{F}=(\alpha,\beta)=\paren{\begin{array}{c}\alpha\\ \beta\end{array}}$\\
$2$-form $fdx\wedge{}dy$ & Scalar Function $f$\\
$\displaystyle{}df=\frac{\partial{}f}{\partial{}x}dx+\frac{\partial{}f}{\partial{}y}dy$ & $\displaystyle\operatorname{grad}F=\paren{\frac{\partial{}f}{\partial{}x},\frac{\partial{}f}{\partial{}y}}=\paren{\frac{\partial}{\partial{}x},\frac{\partial}{\partial{}y}}f=\nabla{}f$\\
$\displaystyle\int_{M^{1\ptxt{-mfd}}}\omega^{1\ptxt{-form}}$ & $\displaystyle\int_{M}\iprod{\vec{F},d\vec{s}}=\int_{M}\iprod{\vec{F},\vec{T}}\,ds$ (where $d\vec{s}=\paren{\begin{array}{cc}dx & dy\end{array}}$)\\
$\displaystyle\int_{M}df=\Delta_{M}f$ & $\displaystyle\int_{M}\iprod{\nabla{}f,\vec{T}}\,ds=\Delta_{m}f$
\end{tabular}
\end{center}

\par\noindent Standard interpretations: $f$ is potential energy, force is $-\nabla{}f$, and work is $\int_{M}\iprod{-\nabla{}f,\vec{T}}\,ds=-\Delta_{m}f$.\n

\par\noindent Suppose $\vec{F}:\R^{2}\to\R^{2}$ is the velocity of a fluid (which could be time-dependent).

\end{document}