\documentclass[10pt,letterpaper]{article}
\usepackage[utf8]{inputenc}
\usepackage[intlimits]{amsmath}
\usepackage{amsfonts}
\usepackage{amssymb}
\usepackage{ragged2e}
\usepackage[letterpaper, margin=1in]{geometry}
\usepackage{graphicx}
\usepackage{cancel}
\usepackage{mathtools}
\usepackage{tabularx}
\usepackage{arydshln}
\usepackage{tensor}
\usepackage{array}
\usepackage{xcolor}
\usepackage[boxed]{algorithm}
\usepackage[noend]{algpseudocode}
\usepackage{listings}
\usepackage{textcomp}
\usepackage[pdf,tmpdir,singlefile]{graphviz}
\usepackage{mathrsfs}
\usepackage{bbm}
\usepackage{tikz}
\usepackage{enumitem}
\usepackage{arydshln}
\usepackage{relsize}
\usepackage{multicol}

%%%%%%%%%%%%%%%%%%%%%%%%%%%%%
% Formatting commands
%%%%%%%%%%%%%%%%%%%%%%%%%%%%%
\newcommand{\n}{\hfill\break}
\newcommand{\up}{\vspace{-\baselineskip}}
\newcommand{\lemma}[1]{\par\noindent\settowidth{\hangindent}{\textbf{Lemma: }}\textbf{Lemma: }#1}
\newcommand{\defn}[1]{\par\noindent\settowidth{\hangindent}{\textbf{Defn: }}\textbf{Defn: }#1\n}
\newcommand{\thm}[1]{\par\noindent\settowidth{\hangindent}{\textbf{Thm: }}\textbf{Thm: }#1\n}
\newcommand{\prop}[1]{\par\noindent\settowidth{\hangindent}{\textbf{Prop: }}\textbf{Prop: }#1\n}
\newcommand{\cor}[1]{\par\noindent\settowidth{\hangindent}{\textbf{Cor: }}\textbf{Cor: }#1\n}
\newcommand{\ex}[1]{\par\noindent\settowidth{\hangindent}{\textbf{Ex: }}\textbf{Ex: }#1\n}
\newcommand{\exer}[1]{\par\noindent\settowidth{\hangindent}{\textbf{Exer: }}\textbf{Exer: }#1\n}
\newcommand{\proven}{\;$\square$\n}
\newcommand{\problem}[1]{\par\noindent{#1}\n}
\newcommand{\problempart}[2]{\par\noindent\indent{}\settowidth{\hangindent}{\textbf{(#1)} \indent{}}\textbf{(#1)} #2\n}
\newcommand{\ptxt}[1]{\textrm{\textnormal{#1}}}
\newcommand{\inlineeq}[1]{\centerline{$\displaystyle #1$}}
\newcommand{\pageline}{\noindent\rule{\textwidth}{0.1pt}}

%%%%%%%%%%%%%%%%%%%%%%%%%%%%%
% Math commands
%%%%%%%%%%%%%%%%%%%%%%%%%%%%%
% Set Theory
\newcommand{\card}[1]{\left|#1\right|}
\newcommand{\set}[1]{\left\{#1\right\}}
\newcommand{\setmid}{\;\middle|\;}
\newcommand{\ps}[1]{\mathcal{P}\left(#1\right)}
\newcommand{\pfinite}[1]{\mathcal{P}^{\ptxt{finite}}\left(#1\right)}
\newcommand{\naturals}{\mathbb{N}}
\newcommand{\N}{\naturals}
\newcommand{\integers}{\mathbb{Z}}
\newcommand{\Z}{\integers}
\newcommand{\rationals}{\mathbb{Q}}
\newcommand{\Q}{\rationals}
\newcommand{\reals}{\mathbb{R}}
\newcommand{\R}{\reals}
\newcommand{\complex}{\mathbb{C}}
\newcommand{\C}{\complex}
\newcommand{\halfPlane}{\mathbb{H}}
\let\H\relax
\newcommand{\H}{\halfPlane}
\newcommand{\comp}{^{\complement}}
\DeclareMathOperator{\Hom}{Hom}
\newcommand{\Ind}{\mathbbm{1}}
\newcommand{\cut}{\setminus}

% Graph Theory
\let\deg\relax
\DeclareMathOperator{\deg}{deg}
\newcommand{\degp}{\ptxt{deg}^{+}}
\newcommand{\degn}{\ptxt{deg}^{-}}
\newcommand{\precdot}{\mathrel{\ooalign{$\prec$\cr\hidewidth\hbox{$\cdot\mkern0.5mu$}\cr}}}
\newcommand{\succdot}{\mathrel{\ooalign{$\cdot\mkern0.5mu$\cr\hidewidth\hbox{$\succ$}\cr\phantom{$\succ$}}}}
\DeclareMathOperator{\cl}{cl}
\DeclareMathOperator{\affdim}{affdim}

% Probability
\newcommand{\Prob}{\mathbb{P}}
\newcommand{\Avg}{\mathbb{E}}

% Standard Math
\newcommand{\inv}{^{-1}}
\newcommand{\abs}[1]{\left|#1\right|}
\newcommand{\ceil}[1]{\left\lceil{}#1\right\rceil{}}
\newcommand{\floor}[1]{\left\lfloor{}#1\right\rfloor{}}
\newcommand{\conj}[1]{\overline{#1}}
\newcommand{\of}{\circ}
\newcommand{\tri}{\triangle}
\newcommand{\inj}{\hookrightarrow}
\newcommand{\surj}{\twoheadrightarrow}
\newcommand{\ndiv}{\nmid}
\renewcommand{\epsilon}{\varepsilon}
\newcommand{\divides}{\mid}
\newcommand{\ndivides}{\nmid}
\DeclareMathOperator{\lcm}{lcm}
\DeclareMathOperator{\sgn}{sgn}
\newcommand{\map}[4]{\!\!\!\begin{array}[t]{rcl}#1 & \!\!\!\!\to & \!\!\!\!#2\\ #3 & \!\!\!\!\mapsto & \!\!\!\!#4\end{array}}
\newcommand{\bigsum}[2]{\smashoperator[lr]{\sum_{\scalebox{#1}{$#2$}}}}

% Linear Algebra
\newcommand{\Id}{\textrm{\textnormal{Id}}}
\newcommand{\im}{\textrm{\textnormal{im}}}
\newcommand{\norm}[1]{\abs{\abs{#1}}}
\newcommand{\tpose}{^{T}}
\newcommand{\iprod}[1]{\left<#1\right>}
\DeclareMathOperator{\trace}{tr}
\newcommand{\chgBasMat}[3]{\!\!\tensor*[_{#1}]{\left[#2\right]}{_{#3}}}
\newcommand{\vecBas}[2]{\tensor*[]{\left[#1\right]}{_{#2}}}
\DeclareMathOperator{\GL}{GL}
\DeclareMathOperator{\Mat}{Mat}
\DeclareMathOperator{\vspan}{span}
\DeclareMathOperator{\rank}{rank}
\newcommand{\V}[1]{\vec{#1}}
\DeclareMathOperator{\proj}{proj}
\DeclareMathOperator{\compProj}{comp}

% Multilinear Algebra
\newcommand{\Lsym}{\L}
\let\L\relax
\DeclareMathOperator{\L}{\mathscr{L}}
\DeclareMathOperator{\A}{\mathcal{A}}
\DeclareMathOperator{\Alt}{Alt}
\DeclareMathOperator{\Sym}{Sym}
\newcommand{\ot}{\otimes}
\newcommand{\ox}{\otimes}
\DeclareMathOperator{\asc}{asc}
\DeclareMathOperator{\asSet}{set}
\DeclareMathOperator{\sort}{sort}
\DeclareMathOperator{\ringA}{\mathring{A}}

% Topology
\newcommand{\closure}[1]{\overline{#1}}
\newcommand{\uball}{\mathcal{U}}
\DeclareMathOperator{\Int}{Int}
\DeclareMathOperator{\Ext}{Ext}
\DeclareMathOperator{\Bd}{Bd}
\DeclareMathOperator{\rInt}{rInt}
\DeclareMathOperator{\ch}{ch}
\DeclareMathOperator{\ah}{ah}
\newcommand{\Tau}{\mathlarger{\mathlarger{\mathlarger{\mathlarger{\tau}}}}}

% Analysis
\DeclareMathOperator{\Graph}{Graph}
\DeclareMathOperator{\epi}{epi}
\DeclareMathOperator{\hypo}{hypo}
\DeclareMathOperator{\supp}{supp}
\newcommand{\lint}[2]{\underset{#1}{\overset{#2}{{\color{black}\underline{{\color{white}\overline{{\color{black}\int}}\color{black}}}}}}}
\newcommand{\uint}[2]{\underset{#1}{\overset{#2}{{\color{white}\underline{{\color{black}\overline{{\color{black}\int}}\color{black}}}}}}}
\newcommand{\alignint}[2]{\underset{#1}{\overset{#2}{{\color{white}\underline{{\color{white}\overline{{\color{black}\int}}\color{black}}}}}}}
\newcommand{\extint}{\ptxt{ext}\int}
\newcommand{\extalignint}[2]{\ptxt{ext}\alignint{#1}{#2}}
\newcommand{\conv}{\ast}

% Proofs
\newcommand{\st}{s.t.}
\newcommand{\unique}{!}
\newcommand{\iffdef}{\overset{\ptxt{def}}{\Leftrightarrow}}
\newcommand{\eqdef}{\overset{\ptxt{def}}{=}}
\newcommand{\eqVertical}{\rotatebox[origin=c]{90}{=}}
\newcommand{\mapsfrom}{\mathrel{\reflectbox{\ensuremath{\mapsto}}}}
\newcommand{\mapsdown}{\rotatebox[origin=c]{-90}{$\mapsto$}\mkern2mu}
\newcommand{\mapsup}{\rotatebox[origin=c]{90}{$\mapsto$}\mkern2mu}

% Brackets
\newcommand{\paren}[1]{\left(#1\right)}
\renewcommand{\brack}[1]{\left[#1\right]}
\renewcommand{\brace}[1]{\left\{#1\right\}}
\newcommand{\ang}[1]{\left<#1\right>}

% Algorithms
\algrenewcommand{\algorithmiccomment}[1]{\hskip 1em \texttt{// #1}}
\algrenewcommand\algorithmicrequire{\textbf{Input:}}
\algrenewcommand\algorithmicensure{\textbf{Output:}}
\newcommand{\parSymbol}{\P}
\renewcommand{\P}{\ptxt{\textbf{P}}}
\newcommand{\NP}{\ptxt{\textbf{NP}}}
\newcommand{\NPC}{\ptxt{\textbf{NP-Complete}}}
\newcommand{\NPH}{\ptxt{\textbf{NP-Hard}}}
\newcommand{\EXP}{\ptxt{\textbf{EXP}}}

%%%%%%%%%%%%%%%%%%%%%%%%%%%%%
% Other commands
%%%%%%%%%%%%%%%%%%%%%%%%%%%%%
\newcommand{\flag}[1]{\textbf{\textcolor{red}{#1}}}

%%%%%%%%%%%%%%%%%%%%%%%%%%%%%
% Make l's curvy in math environments
%%%%%%%%%%%%%%%%%%%%%%%%%%%%%
\mathcode`l="8000
\begingroup
\makeatletter
\lccode`\~=`\l
\DeclareMathSymbol{\lsb@l}{\mathalpha}{letters}{`l}
\lowercase{\gdef~{\ifnum\the\mathgroup=\m@ne \ell \else \lsb@l \fi}}%
\endgroup

\newcommand{\B}{
    \begin{tikzpicture}
    \filldraw [fill=red, draw=black] (0, 0) rectangle (0.37, 0.45);
    \draw [line width=0.5mm, white ] (0.1,0.08) -- (0.1,0.38);
    \draw[line width=0.5mm, white ] (0.1, 0.35) .. controls (0.2, 0.35) and (0.4, 0.2625) .. (0.1, 0.225);
    \draw[line width=0.5mm, white ] (0.1, 0.225) .. controls (0.2, 0.225) and (0.4, 0.1625) .. (0.1, 0.1);
    \end{tikzpicture}
}

\author{Professor David Barrett\\ \small\textit{Transcribed by Thomas Cohn}}
\title{Proving Facts about the Wedge Product}
\date{1/28/19} % Can also use \today

\begin{document}
\maketitle
\setlength\RaggedRightParindent{\parindent}
\RaggedRight

\par\noindent Recall:\n
$I=(i_{1},\ldots,i_{k})$ and $J=(j_{1},\ldots,j_{k})$ $k$-forms. $\phi_{I}(\vec{a_{J}})=\left\{\begin{array}{ll}1 & \quad{}I=J\\ 0 & \quad{}I\ne{}J\end{array}\right.$ for $\vec{a_{J}}=(\vec{a_{j_{1}}},\ldots,\vec{a_{j_{k}}})$, where the $\vec{a_{j}}$ are a basis for $V$.\n
$\phi_{I}\in\L^{k}(V)$.\n
$\sgn(I,J)\eqdef(-1)^{\ptxt{\# of transpositions to sort }(I,J)}$.\n

\par\noindent $\psi_{i}=\phi_{i}\in\L^{1}(V)=\A^{1}(V)=V^{*}$. For $I\asc$, $\psi_{I}=\sum_{\sigma\in{}S_{k}}\sgn\sigma\cdot\phi_{I_{\sigma}}\in\A^{k}$.\n

\par\noindent Wedge product $\map{\A^{k}(V)\times\A^{k}(V)}{\A^{k+l}(V)}{(f,g)}{f\wedge{}g}$ where
\[
\paren{\sum_{I\asc}\alpha_{I}\psi_{I}}\wedge\paren{\sum_{J\asc}\beta_{J}\psi_{J}}=\bigsum{0.75}{\begin{array}{c}I\asc\\ J\asc\\ \ptxt{no duplications}\end{array}}\alpha_{I}\beta_{J}\sgn(I,J)
\]

\thm{\begin{enumerate}[label=(\alph*)]
	\item $f\wedge{}g$ linear in $f$, linear in $g$.\n
	Proof: clear from definition.
	\item $(f\wedge{}g)\wedge{}h=f\wedge(g\wedge{}h)$.
	\item $g\wedge{}f=(-1)^{kl}f\wedge{}g$.
	\item $\psi_{I}=\psi_{i_{1}}\wedge\cdots\wedge\psi_{i_{k}}$.\n
	Proof: clear from definition.
	\item $T^{*}(f\wedge{}g)=T^{*}f\wedge{}T^{*}g$.
\end{enumerate}}\up

\par\noindent Special case: $\dim{}V=3$.\n
$(\alpha\psi_{1}+\beta\psi_{2}+\gamma\psi_{3})\wedge(\tilde{\alpha}\psi_{4}+\tilde{\beta}\psi_{5}+\tilde{\gamma}\psi_{6})\wedge(\hat{\alpha}\psi_{7}+\hat{\beta}\psi_{8}+\hat{\gamma}\psi_{9})=\cdots=\det\paren{\begin{array}{ccc}\alpha & \beta & \gamma\\ \tilde{\alpha} & \tilde{\beta} & \tilde{\gamma}\\ \hat{\alpha} & \hat{\beta} & \hat{\gamma}\end{array}}$.\up\n
Wedge of $3$ $(n-1)$-tensors yields the determinant.\n

\par\noindent $(\alpha\psi_{1}+\beta\psi_{2}+\gamma\psi_{3})\wedge(A\psi_{(2,3)}\pm{}B\psi_{(1,3)}+C\psi_{(1,2)})=\cdots=(\alpha{}A\pm\beta{}B+\gamma{}C)\psi_{(1,2,3)}$.\n
Wedge of $1$-tensor and alternating $(n-1)$-tensor yields the dot product (with minus signs).\n

\par\noindent $(\alpha\psi_{1}+\beta\psi_{2}+\gamma\psi_{3})\wedge(\tilde{\alpha}\psi_{4}+\tilde{\beta}\psi_{5}+\tilde{\gamma}\psi_{6})=(\alpha\tilde{\beta}-\beta\tilde{\alpha})\psi_{(1,2)}+(\beta\tilde{\gamma}-\gamma\tilde{\beta})\psi_{(2,3)}+(\alpha\tilde{\gamma}-\gamma\tilde{\alpha})\psi_{(1,3)}$.\n
Wedge of $2$ $(n-1)$-tensors yields the cross product.\n

\par\noindent Proof of (c): Examine the key special case where $f=\psi_{I}$ and $g=\psi_{J}$ are basis elements.\n
Then $\sgn(J,I)=(-1)^{kl}\sgn(I,J)$, so $\psi_{J}\wedge\psi_{I}=(-1)^{kl}\psi_{I}\wedge\psi_{J}$. The general case is left as an exercise.\proven

\par\noindent Proof of (b): Examine the key special case where $f=\psi_{I}$, $g=\psi_{J}$, and $h=\psi_{K}$ are basis elements.\n
Then $\sgn(I,J)\sgn(\sort(I,J),K)=\sgn(I,\sort(J,K))\sgn(J,K)=\sgn(I,J,K)$.\n
So $(\psi_{I}\wedge\psi_{J})\wedge\psi_{K}=\psi_{I}\wedge(\psi_{J}\wedge\psi_{K})$. The general case is left as an exercise.\proven

\par\noindent Proof of (e): We need a basis for $V,W$. Related questions: can we provide a ``basis-free'' definition of $\wedge$? Does $\wedge$ depend on the choice of basis?\n

\defn{$\ringA:\map{\L^{k}(V)}{\A^{k}(V)}{f}{\sum_{\sigma}\sgn(\sigma)f^{\sigma}}$.}

\exer{$\ringA{}f\in\A^{k}(V)$ (see p238)}

\par\noindent Note:
\begin{enumerate}[label=\alph*),topsep=0pt,itemsep=0pt]
	\item $\ringA$ doesn't use a basis
	\item $I\asc\Rightarrow\ringA\phi_{I}=\psi_{I}$
	\item $\#(I_{\asSet})<k\Rightarrow\ringA\phi_{I}=0$
	\item $f\in\A^{k}(V)\Rightarrow\ringA{}f=k!f$
\end{enumerate}

\vspace{15pt}
\prop{$f\wedge{}g=\frac{1}{k!l!}\ringA(f\ot{}g)$}

\cor{Resultant definition of $\wedge$ does not depend on the choice of basis.}

\par\noindent Proof of (e), assuming Prop:\up
\begin{align*}
	T^{*}(f\wedge{}g) & =\frac{1}{k!l!}T^{*}\paren{\sum_{\sigma\in{}S_{k+l}}\sgn(\sigma)(f\ox{}g)^{\sigma}}\\
	& =\frac{1}{k!l!}\sum_{\sigma\in{}S_{k+l}}\sgn(\sigma)(T^{*}(f\ox{}g))^{\sigma}\\
	& =\frac{1}{k!l!}\sum_{\sigma\in{}S_{k+l}}\sgn(\sigma)(T^{*}f\ox{}T^{*}g)^{\sigma}\\
	& =\frac{1}{k!l!}\ringA(T^{*}f\ox{}T^{*}g)\\
	& =T^{*}f\wedge{}T^{*}g\qquad\square
\end{align*}

\par\noindent Proof of Prop: Examine the key special case where $f=\psi_{I}$ and $g=\psi_{J}$ with $I,J\asc$.\n
It is enough to show $(\psi_{I}\wedge\psi_{J})(\vec{a_{S}})=\frac{1}{k!l!}\ringA(\psi_{I}\ox\psi_{J})(\vec{a}_{S})$ for $S\asc$ $(k+l)$-tuple.\n

\par\noindent $LHS=\left\{\begin{array}{ll}0 & \quad{}I_{\asSet}\cup{}J_{\asSet}\ne{}S_{\asSet}\\ \sgn(I,J) & \quad\ptxt{o/w}\end{array}\right.$\n
$\displaystyle{}RHS=\frac{1}{k!l!}\sum_{\sigma\in{}S_{k+l}}\underbrace{\sgn(\sigma)(\psi_{I}\ox\psi_{J})(\vec{a}_{S_{\sigma}})}_{*}$.\n
$\displaystyle{}*=\left\{\begin{array}{ll}0 & \quad{}I_{\asSet}\ne(S_{\sigma}')_{\asSet}\lor{}J_{\asSet}\ne(S_{\sigma}'')_{\asSet}\\ \sgn\sigma\cdot\sgn\sigma'\cdot\sgn\sigma''\cdots\overset{\dagger}{=}\sgn(I,J) & \quad\ptxt{o/w}\end{array}\right.$\n
$\dagger$ because it happens $k!l!$ times.\n

\par\noindent Thus, $RHS=LHS$.\proven

\par\noindent We've now learned the basics of \underline{exterior algebra}. It was developed by Grassman in the mid 1800s, with the goal of studying subspaces of vector spaces.\n

\par\noindent We need to move on to \underline{exterior calculus}. It was developed by Elie Catan from 1869-1951.

\end{document}