\documentclass[10pt,letterpaper]{article}
\usepackage[utf8]{inputenc}
\usepackage[intlimits]{amsmath}
\usepackage{amsfonts}
\usepackage{amssymb}
\usepackage{ragged2e}
\usepackage[letterpaper, margin=1in]{geometry}
\usepackage{graphicx}
\usepackage{cancel}
\usepackage{mathtools}
\usepackage{tabularx}
\usepackage{arydshln}
\usepackage{tensor}
\usepackage{array}
\usepackage{xcolor}
\usepackage[boxed]{algorithm}
\usepackage[noend]{algpseudocode}
\usepackage{listings}
\usepackage{textcomp}
\usepackage[pdf,tmpdir,singlefile]{graphviz}
\usepackage{mathrsfs}
\usepackage{bbm}
\usepackage{tikz}
\usepackage{enumitem}
\usepackage{arydshln}
\usepackage{relsize}

%%%%%%%%%%%%%%%%%%%%%%%%%%%%%
% Formatting commands
%%%%%%%%%%%%%%%%%%%%%%%%%%%%%
\newcommand{\n}{\hfill\break}
\newcommand{\up}{\vspace{-\baselineskip}}
\newcommand{\lemma}[1]{\par\noindent\settowidth{\hangindent}{\textbf{Lemma: }}\textbf{Lemma: }#1}
\newcommand{\defn}[1]{\par\noindent\settowidth{\hangindent}{\textbf{Defn: }}\textbf{Defn: }#1\n}
\newcommand{\thm}[1]{\par\noindent\settowidth{\hangindent}{\textbf{Thm: }}\textbf{Thm: }#1\n}
\newcommand{\prop}[1]{\par\noindent\settowidth{\hangindent}{\textbf{Prop: }}\textbf{Prop: }#1\n}
\newcommand{\cor}[1]{\par\noindent\settowidth{\hangindent}{\textbf{Cor: }}\textbf{Cor: }#1\n}
\newcommand{\ex}[1]{\par\noindent\settowidth{\hangindent}{\textbf{Ex: }}\textbf{Ex: }#1\n}
\newcommand{\exer}[1]{\par\noindent\settowidth{\hangindent}{\textbf{Exer: }}\textbf{Exer: }#1\n}
\newcommand{\proven}{\;$\square$\n}
\newcommand{\problem}[1]{\par\noindent{#1}\n}
\newcommand{\problempart}[2]{\par\noindent\indent{}\settowidth{\hangindent}{\textbf{(#1)} \indent{}}\textbf{(#1)} #2\n}
\newcommand{\ptxt}[1]{\textrm{\textnormal{#1}}}
\newcommand{\inlineeq}[1]{\centerline{$\displaystyle #1$}}
\newcommand{\pageline}{\noindent\rule{\textwidth}{0.1pt}}

%%%%%%%%%%%%%%%%%%%%%%%%%%%%%
% Math commands
%%%%%%%%%%%%%%%%%%%%%%%%%%%%%
% Set Theory
\newcommand{\card}[1]{\left|#1\right|}
\newcommand{\set}[1]{\left\{#1\right\}}
\newcommand{\setmid}{\;\middle|\;}
\newcommand{\ps}[1]{\mathcal{P}\left(#1\right)}
\newcommand{\pfinite}[1]{\mathcal{P}^{\ptxt{finite}}\left(#1\right)}
\newcommand{\naturals}{\mathbb{N}}
\newcommand{\N}{\naturals}
\newcommand{\integers}{\mathbb{Z}}
\newcommand{\Z}{\integers}
\newcommand{\rationals}{\mathbb{Q}}
\newcommand{\Q}{\rationals}
\newcommand{\reals}{\mathbb{R}}
\newcommand{\R}{\reals}
\newcommand{\complex}{\mathbb{C}}
\newcommand{\C}{\complex}
\newcommand{\halfPlane}{\mathbb{H}}
\let\H\relax
\newcommand{\H}{\halfPlane}
\newcommand{\comp}{^{\complement}}
\DeclareMathOperator{\Hom}{Hom}
\newcommand{\Ind}{\mathbbm{1}}
\newcommand{\cut}{\setminus}

% Graph Theory
\let\deg\relax
\DeclareMathOperator{\deg}{deg}
\newcommand{\degp}{\ptxt{deg}^{+}}
\newcommand{\degn}{\ptxt{deg}^{-}}
\newcommand{\precdot}{\mathrel{\ooalign{$\prec$\cr\hidewidth\hbox{$\cdot\mkern0.5mu$}\cr}}}
\newcommand{\succdot}{\mathrel{\ooalign{$\cdot\mkern0.5mu$\cr\hidewidth\hbox{$\succ$}\cr\phantom{$\succ$}}}}
\DeclareMathOperator{\cl}{cl}
\DeclareMathOperator{\affdim}{affdim}

% Probability
\newcommand{\Prob}{\mathbb{P}}
\newcommand{\Avg}{\mathbb{E}}

% Standard Math
\newcommand{\inv}{^{-1}}
\newcommand{\abs}[1]{\left|#1\right|}
\newcommand{\ceil}[1]{\left\lceil{}#1\right\rceil{}}
\newcommand{\floor}[1]{\left\lfloor{}#1\right\rfloor{}}
\newcommand{\conj}[1]{\overline{#1}}
\newcommand{\of}{\circ}
\newcommand{\tri}{\triangle}
\newcommand{\inj}{\hookrightarrow}
\newcommand{\surj}{\twoheadrightarrow}
\newcommand{\mapsfrom}{\mathrel{\reflectbox{\ensuremath{\mapsto}}}}
\newcommand{\mapsdown}{\rotatebox[origin=c]{-90}{$\mapsto$}\mkern2mu}
\newcommand{\mapsup}{\rotatebox[origin=c]{90}{$\mapsto$}\mkern2mu}
\newcommand{\ndiv}{\nmid}
\renewcommand{\epsilon}{\varepsilon}
\newcommand{\divides}{\mid}
\newcommand{\ndivides}{\nmid}
\DeclareMathOperator{\lcm}{lcm}
\newcommand{\eqVertical}{\rotatebox[origin=c]{90}{=}}
\DeclareMathOperator{\sgn}{sgn}

% Linear Algebra
\newcommand{\Id}{\textrm{\textnormal{Id}}}
\newcommand{\im}{\textrm{\textnormal{im}}}
\newcommand{\norm}[1]{\abs{\abs{#1}}}
\newcommand{\tpose}{^{T}}
\newcommand{\iprod}[1]{\left<#1\right>}
\DeclareMathOperator{\trace}{tr}
\newcommand{\chgBasMat}[3]{\!\!\tensor*[_{#1}]{\left[#2\right]}{_{#3}}}
\newcommand{\vecBas}[2]{\tensor*[]{\left[#1\right]}{_{#2}}}
\DeclareMathOperator{\GL}{GL}
\DeclareMathOperator{\Mat}{Mat}
\DeclareMathOperator{\vspan}{span}
\DeclareMathOperator{\rank}{rank}
\newcommand{\V}[1]{\vec{#1}}

% Topology
\newcommand{\closure}[1]{\overline{#1}}
\newcommand{\uball}{\mathcal{U}}
\DeclareMathOperator{\Int}{Int}
\DeclareMathOperator{\Ext}{Ext}
\DeclareMathOperator{\Bd}{Bd}
\DeclareMathOperator{\rInt}{rInt}
\DeclareMathOperator{\ch}{ch}
\DeclareMathOperator{\ah}{ah}
\newcommand{\Tau}{\mathlarger{\mathlarger{\mathlarger{\mathlarger{\tau}}}}}

% Analysis
\DeclareMathOperator{\Graph}{Graph}
\DeclareMathOperator{\epi}{epi}
\DeclareMathOperator{\hypo}{hypo}
\DeclareMathOperator{\supp}{supp}
\newcommand{\lint}[2]{\underset{#1}{\overset{#2}{{\color{black}\underline{{\color{white}\overline{{\color{black}\int}}\color{black}}}}}}}
\newcommand{\uint}[2]{\underset{#1}{\overset{#2}{{\color{white}\underline{{\color{black}\overline{{\color{black}\int}}\color{black}}}}}}}
\newcommand{\alignint}[2]{\underset{#1}{\overset{#2}{{\color{white}\underline{{\color{white}\overline{{\color{black}\int}}\color{black}}}}}}}
\newcommand{\extint}{\ptxt{ext}\int}
\newcommand{\extalignint}[2]{\ptxt{ext}\alignint{#1}{#2}}
\newcommand{\conv}{\ast}

% Proofs
\newcommand{\st}{s.t.}
\newcommand{\unique}{!}
\newcommand{\iffdef}{\overset{\ptxt{def}}{\Leftrightarrow}}
\newcommand{\eqdef}{\overset{\ptxt{def}}{=}}

% Brackets
\newcommand{\paren}[1]{\left(#1\right)}
\renewcommand{\brack}[1]{\left[#1\right]}
\renewcommand{\brace}[1]{\left\{#1\right\}}
\newcommand{\ang}[1]{\left<#1\right>}

% Algorithms
\algrenewcommand{\algorithmiccomment}[1]{\hskip 1em \texttt{// #1}}
\algrenewcommand\algorithmicrequire{\textbf{Input:}}
\algrenewcommand\algorithmicensure{\textbf{Output:}}
\newcommand{\parSymbol}{\P}
\renewcommand{\P}{\ptxt{\textbf{P}}}
\newcommand{\NP}{\ptxt{\textbf{NP}}}
\newcommand{\NPC}{\ptxt{\textbf{NP-Complete}}}
\newcommand{\NPH}{\ptxt{\textbf{NP-Hard}}}
\newcommand{\EXP}{\ptxt{\textbf{EXP}}}

%%%%%%%%%%%%%%%%%%%%%%%%%%%%%
% Other commands
%%%%%%%%%%%%%%%%%%%%%%%%%%%%%
\newcommand{\flag}[1]{\textbf{\textcolor{red}{#1}}}

%%%%%%%%%%%%%%%%%%%%%%%%%%%%%
% Make l's curvy in math environments
%%%%%%%%%%%%%%%%%%%%%%%%%%%%%
\mathcode`l="8000
\begingroup
\makeatletter
\lccode`\~=`\l
\DeclareMathSymbol{\lsb@l}{\mathalpha}{letters}{`l}
\lowercase{\gdef~{\ifnum\the\mathgroup=\m@ne \ell \else \lsb@l \fi}}%
\endgroup

\newcommand{\B}{
    \begin{tikzpicture}
    \filldraw [fill=red, draw=black] (0, 0) rectangle (0.37, 0.45);
    \draw [line width=0.5mm, white ] (0.1,0.08) -- (0.1,0.38);
    \draw[line width=0.5mm, white ] (0.1, 0.35) .. controls (0.2, 0.35) and (0.4, 0.2625) .. (0.1, 0.225);
    \draw[line width=0.5mm, white ] (0.1, 0.225) .. controls (0.2, 0.225) and (0.4, 0.1625) .. (0.1, 0.1);
    \end{tikzpicture}
}

\author{Professor David Barrett\\ \small\textit{Transcribed by Thomas Cohn}}
\title{Integral Manifolds and Differential Equations}
\date{1/14/19} % Can also use \today

\begin{document}
\maketitle
\setlength\RaggedRightParindent{\parindent}
\RaggedRight

\par\noindent Recall from Friday/HW 1 \#3:

\defn{$M\subset{}A$ is an \underline{integral manifold} for $\omega$ $1$-form on $A^{\ptxt{open}}\subset\R^{n}$ when any of the following conditions are true (TFAE):
\begin{enumerate}[label=(\alph*), leftmargin=4\parindent]
	\item $\Tau_{\vec{p}}M\subset\ker\omega(\vec{p})$, $\forall\vec{p}\in{}M$
	\item $\alpha^{*}\omega=0$, $\forall\alpha$ coordinate patch for $M$
	\item $\int_{C}\omega=0$, $\forall{}C^{\ptxt{$1$-mfd}}\subset{}M$
\end{enumerate}\up}

\ex{Consider $n=2$...\n
$f\in{}C^{1}(A,\R)$, $df\ne{}0$ on $A$.\n
Then each level set $f\inv(c)$ is a $1$-mfd-wob.\n
Then each level set of $f$ is an integral manifold for $df$.\n
Proof: $\alpha^{*}(df)\underset{\ptxt{15}}{=}d(\alpha^{*}f)\underset{\ptxt{4}}{=}d(f\of\alpha)=d(c)\underset{\ptxt{12}}=0$\n
\n
Suppose $\omega=u(x,y)dx+v(x,y)dy$ on $A^{\ptxt{open}}\subset\R^{2}$ with $v$ non-vanishing.\n
Given $M$ is a $1$-mfd, use a graph paramterization $\begin{array}{rcl}\alpha:(a,b) & \!\!\!\!\to & \!\!\!\!M\\ x & \!\!\!\!\mapsto & \!\!\!\!(x,f(x))\end{array}$\n
\n
All other coord patches $\beta$ satisfy $\beta=\alpha\of\underbrace{(\alpha\inv\of\beta)}_{\ptxt{transition map}}=\alpha\of\gamma$ for $\gamma$ $C^{1}$ diffeomorphism.\n
That is, $\beta^{*}\omega\overset{\gamma}{=}\gamma^{*}(\alpha^{*}\omega)$.\n
\n
Thus, $M$ is an integral manifold for $\omega$ if and only if $\alpha^{*}\omega=0$.\n
$\alpha^{*}\omega=u(x,f(x))dx+v(x,f(x))d(f(x))=u(x,f(x))d+v(x,f(x))f'(x)dx$.\n
This is $0$ if and only if $f'(x)=\frac{-u(x,f(x))}{v(x,f(x))}$. This is a differential equation for $f$.}

\par\noindent Suppose further that $u,v\in{}C^{1}$, so $-\frac{u(x,y)}{v(x,y)}$ is $C^{1}$. Consider $(0,y_{0})\in{}A$.\n
Claim: $\exists\Phi\in{}C^{1}(\R^{2},\R)$ with $D\Phi:\R^{2}\to\R^{2}_{\ptxt{row}}=\Hom(\R^{2},\R)$ bounded and $\Phi(x,y)=-\frac{u(x,y)}{v(x,y)}$ for $(x,y)$ in a neighborhood of $(0,y_{0})$.\n
Proof: Pick a $\Psi\in{}C^{\infty}(\R^{2},\R)$ with $\Psi\equiv{}1$ in a neighborhood around $(0,y_{0})$, and $\supp\Psi$ is a compact subset of $A$ (see notes 11/21 for why we can do this).\n
Set $\Phi(x,y)=\left\{\begin{array}{ll}-\frac{u(x,y)}{v(x,y)}\cdot\Psi(x,y) & (x,y)\in{}A\\ 0 & (x,y)\not\in{}A\end{array}\right.$\n

\exer{Check that this works.}

\par\noindent 395 HW 3 \#2 $\Rightarrow$ $\Phi$ is Lipschitz on $\R^{2}$ $\Rightarrow$ $\Phi$ is partial-Lipschitz on $\R^{2}$\n
395 HW 10 \#6 $\Rightarrow$ $\exists\tilde{\epsilon}>0$ \st{} $f'(x)=\Phi(x,f(x))$ with $f(0)=y_{0}$ has a unique solution for $x\in(-\tilde{\epsilon},\tilde{\epsilon})$ (perhaps shrinking $\tilde{\epsilon}$ if necessary).\n

\par\noindent Result: $\exists\epsilon>0$ such that $f'(x)=-\frac{u(x,f(x))}{v(x,f(x))}$ has a unique solution for $x\in(-\epsilon,\epsilon)$. So $\exists\unique$ local integral curve for $\omega$ passing through $(0,y_{0})$.\n

\par\noindent Claim: Get the same result based at $(x_{0},y_{0})$.\n
Proof: Use a translation in the $x$-direction.\n

\par\noindent 395 HW 11 \#6: $\omega=x^{1/3}dx-dy$ is a $C^{0}$ but not $C^{1}$ $1$-form. $\omega$ does not have unique solutions.\n

\par\noindent Conversely, given a differential equation $f'(x)=\Phi(x,f(x))$ $\star$\n
Then graphs of solutions of ($\star$) are integral curves for $\omega=-B(x,y)\Phi(x,y)dx+B(x,y)dy$.\n
I.e. ``$0=-B(x,y)\Phi(x,y)+B(x,y)\frac{dy}{dx}$''.\n
Suppose we can choose non-zero $B$ such that $B\omega$ is exact, i.e., $B\omega=dg$ for some $g$.\n

\defn{Then $B$ is an \underline{integrating factor} for $-\Phi(x,y)dx+dy$.}

\par\noindent From earlier in lecture, the level curves $g\inv(c)$ are integral manifolds for $dg$, which are integral manifolds for $-\Phi(x,y)dx+dy$, which are graphs of solutions of ($\star$).\n

\par\noindent Conversely, $f$ solves ($\star$) $\Rightarrow$ $\frac{d}{dx}g(x,f(x))=\paren{\begin{array}{cc}-B(x,f(x)) & \Phi(x,f(x))\end{array}}\paren{\begin{array}{c}1\\ \Phi(x,f(x))\end{array}}=0$.\n
So $g(x,f(x))$ is (locally) constant!\n

\par\noindent \textbf{Good News 1:} Such a $B$ always exists!\n
\textbf{Good News 2:} Looking for such a $B$ is often a useful approach to solving ($\star$)!\n
\textbf{Bad News:} It's not always easier to find $B$ than to solve ($\star$).\n

\par\noindent Two important classes of examples:\n

\par\noindent{\Huge 1}\settowidth{\hangindent}{{\Huge 1}} $f'(x)=\beta(f(x))$ ($\star\star$), an ``autonomous differential equation''. Solutions are integral curves for\n
$-\beta(y)dx-dy$. Use integrating factor $B(x,y)=\frac{1}{\beta(y)}$ (assume for now that $\beta$ doesn't vanish).\n
Solutions: integral cuves for $-dx+\frac{dy}{\beta(y)}=d(-x-\int\frac{dy}{\beta(y)})$.\n
Solutions of ($\star\star$) satisfy $-x+\int\frac{dy}{\beta(y)}=C$, i.e., $\int\frac{dy}{\beta(y)}=x+C$.\n

\exer{Try to solve for $y$ using implicit function theorem if there's no nice closed form solution as a function of $y$.}

\end{document}