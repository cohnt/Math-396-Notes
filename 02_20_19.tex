\documentclass[10pt,letterpaper]{article}
\usepackage[utf8]{inputenc}
\usepackage[intlimits]{amsmath}
\usepackage{amsfonts}
\usepackage{amssymb}
\usepackage{ragged2e}
\usepackage[letterpaper, margin=1in]{geometry}
\usepackage{graphicx}
\usepackage{cancel}
\usepackage{mathtools}
\usepackage{tabularx}
\usepackage{arydshln}
\usepackage{tensor}
\usepackage{array}
\usepackage{xcolor}
\usepackage[boxed]{algorithm}
\usepackage[noend]{algpseudocode}
\usepackage{listings}
\usepackage{textcomp}
\usepackage[pdf,tmpdir,singlefile]{graphviz}
\usepackage{mathrsfs}
\usepackage{bbm}
\usepackage{tikz}
\usepackage{enumitem}
\usepackage{arydshln}
\usepackage{relsize}
\usepackage{multicol}
\usepackage{tikz-cd}

%%%%%%%%%%%%%%%%%%%%%%%%%%%%%
% Formatting commands
%%%%%%%%%%%%%%%%%%%%%%%%%%%%%
\newcommand{\n}{\hfill\break}
\newcommand{\up}{\vspace{-\baselineskip}}
\newcommand{\lemma}[1]{\par\noindent\settowidth{\hangindent}{\textbf{Lemma: }}\textbf{Lemma: }#1}
\newcommand{\defn}[1]{\par\noindent\settowidth{\hangindent}{\textbf{Defn: }}\textbf{Defn: }#1\n}
\newcommand{\thm}[1]{\par\noindent\settowidth{\hangindent}{\textbf{Thm: }}\textbf{Thm: }#1\n}
\newcommand{\prop}[1]{\par\noindent\settowidth{\hangindent}{\textbf{Prop: }}\textbf{Prop: }#1\n}
\newcommand{\cor}[1]{\par\noindent\settowidth{\hangindent}{\textbf{Cor: }}\textbf{Cor: }#1\n}
\newcommand{\ex}[1]{\par\noindent\settowidth{\hangindent}{\textbf{Ex: }}\textbf{Ex: }#1\n}
\newcommand{\exer}[1]{\par\noindent\settowidth{\hangindent}{\textbf{Exer: }}\textbf{Exer: }#1\n}
\newcommand{\proven}{\;$\square$\n}
\newcommand{\problem}[1]{\par\noindent{#1}\n}
\newcommand{\problempart}[2]{\par\noindent\indent{}\settowidth{\hangindent}{\textbf{(#1)} \indent{}}\textbf{(#1)} #2\n}
\newcommand{\ptxt}[1]{\textrm{\textnormal{#1}}}
\newcommand{\inlineeq}[1]{\centerline{$\displaystyle #1$}}
\newcommand{\pageline}{\noindent\rule{\textwidth}{0.1pt}}

%%%%%%%%%%%%%%%%%%%%%%%%%%%%%
% Math commands
%%%%%%%%%%%%%%%%%%%%%%%%%%%%%
% Set Theory
\newcommand{\card}[1]{\left|#1\right|}
\newcommand{\set}[1]{\left\{#1\right\}}
\newcommand{\setmid}{\;\middle|\;}
\newcommand{\ps}[1]{\mathcal{P}\left(#1\right)}
\newcommand{\pfinite}[1]{\mathcal{P}^{\ptxt{finite}}\left(#1\right)}
\newcommand{\naturals}{\mathbb{N}}
\newcommand{\N}{\naturals}
\newcommand{\integers}{\mathbb{Z}}
\newcommand{\Z}{\integers}
\newcommand{\rationals}{\mathbb{Q}}
\newcommand{\Q}{\rationals}
\newcommand{\reals}{\mathbb{R}}
\newcommand{\R}{\reals}
\newcommand{\complex}{\mathbb{C}}
\newcommand{\C}{\complex}
\newcommand{\halfPlane}{\mathbb{H}}
\let\H\relax
\newcommand{\H}{\halfPlane}
\newcommand{\comp}{^{\complement}}
\DeclareMathOperator{\Hom}{Hom}
\newcommand{\Ind}{\mathbbm{1}}
\newcommand{\cut}{\setminus}
\DeclareMathOperator{\elem}{elem}

% Graph Theory
\let\deg\relax
\DeclareMathOperator{\deg}{deg}
\newcommand{\degp}{\ptxt{deg}^{+}}
\newcommand{\degn}{\ptxt{deg}^{-}}
\newcommand{\precdot}{\mathrel{\ooalign{$\prec$\cr\hidewidth\hbox{$\cdot\mkern0.5mu$}\cr}}}
\newcommand{\succdot}{\mathrel{\ooalign{$\cdot\mkern0.5mu$\cr\hidewidth\hbox{$\succ$}\cr\phantom{$\succ$}}}}
\DeclareMathOperator{\cl}{cl}
\DeclareMathOperator{\affdim}{affdim}

% Probability
\newcommand{\Prob}{\mathbb{P}}
\newcommand{\Avg}{\mathbb{E}}
\DeclareMathOperator{\Var}{Var}
\DeclareMathOperator{\cov}{cov}

% Standard Math
\newcommand{\inv}{^{-1}}
\newcommand{\abs}[1]{\left|#1\right|}
\newcommand{\ceil}[1]{\left\lceil{}#1\right\rceil{}}
\newcommand{\floor}[1]{\left\lfloor{}#1\right\rfloor{}}
\newcommand{\conj}[1]{\overline{#1}}
\newcommand{\of}{\circ}
\newcommand{\tri}{\triangle}
\newcommand{\inj}{\hookrightarrow}
\newcommand{\surj}{\twoheadrightarrow}
\newcommand{\ndiv}{\nmid}
\renewcommand{\epsilon}{\varepsilon}
\newcommand{\divides}{\mid}
\newcommand{\ndivides}{\nmid}
\DeclareMathOperator{\lcm}{lcm}
\DeclareMathOperator{\sgn}{sgn}
\newcommand{\map}[4]{\!\!\!\begin{array}[t]{rcl}#1 & \!\!\!\!\to & \!\!\!\!#2\\ #3 & \!\!\!\!\mapsto & \!\!\!\!#4\end{array}}
\newcommand{\bigsum}[2]{\smashoperator[lr]{\sum_{\scalebox{#1}{$#2$}}}}

% Linear Algebra
\newcommand{\Id}{\textrm{\textnormal{Id}}}
\newcommand{\im}{\textrm{\textnormal{im}}}
\newcommand{\norm}[1]{\abs{\abs{#1}}}
\newcommand{\tpose}{^{T}}
\newcommand{\iprod}[1]{\left<#1\right>}
\DeclareMathOperator{\trace}{tr}
\newcommand{\chgBasMat}[3]{\!\!\tensor*[_{#1}]{\left[#2\right]}{_{#3}}}
\newcommand{\vecBas}[2]{\tensor*[]{\left[#1\right]}{_{#2}}}
\DeclareMathOperator{\GL}{GL}
\DeclareMathOperator{\Mat}{Mat}
\DeclareMathOperator{\vspan}{span}
\DeclareMathOperator{\rank}{rank}
\newcommand{\V}[1]{\vec{#1}}
\DeclareMathOperator{\proj}{proj}
\DeclareMathOperator{\compProj}{comp}
\DeclareMathOperator{\row}{row}

% Multilinear Algebra
\newcommand{\Lsym}{\L}
\let\L\relax
\DeclareMathOperator{\L}{\mathscr{L}}
\DeclareMathOperator{\A}{\mathcal{A}}
\DeclareMathOperator{\Alt}{Alt}
\DeclareMathOperator{\Sym}{Sym}
\newcommand{\ot}{\otimes}
\newcommand{\ox}{\otimes}
\DeclareMathOperator{\asc}{asc}
\DeclareMathOperator{\asSet}{set}
\DeclareMathOperator{\sort}{sort}
\DeclareMathOperator{\ringA}{\mathring{A}}

% Topology
\newcommand{\closure}[1]{\overline{#1}}
\newcommand{\uball}{\mathcal{U}}
\DeclareMathOperator{\Int}{Int}
\DeclareMathOperator{\Ext}{Ext}
\DeclareMathOperator{\Bd}{Bd}
\DeclareMathOperator{\rInt}{rInt}
\DeclareMathOperator{\ch}{ch}
\DeclareMathOperator{\ah}{ah}
\newcommand{\Tau}{\mathlarger{\mathlarger{\mathlarger{\mathlarger{\tau}}}}}

% Analysis
\DeclareMathOperator{\Graph}{Graph}
\DeclareMathOperator{\epi}{epi}
\DeclareMathOperator{\hypo}{hypo}
\DeclareMathOperator{\supp}{supp}
\newcommand{\lint}[2]{\underset{#1}{\overset{#2}{{\color{black}\underline{{\color{white}\overline{{\color{black}\int}}\color{black}}}}}}}
\newcommand{\uint}[2]{\underset{#1}{\overset{#2}{{\color{white}\underline{{\color{black}\overline{{\color{black}\int}}\color{black}}}}}}}
\newcommand{\alignint}[2]{\underset{#1}{\overset{#2}{{\color{white}\underline{{\color{white}\overline{{\color{black}\int}}\color{black}}}}}}}
\newcommand{\extint}{\ptxt{ext}\int}
\newcommand{\extalignint}[2]{\ptxt{ext}\alignint{#1}{#2}}
\newcommand{\conv}{\ast}
\newcommand{\pd}[2]{\frac{\partial{}#1}{\partial{}#2}}
\newcommand{\del}{\nabla}
\DeclareMathOperator{\grad}{grad}
\DeclareMathOperator{\curl}{curl}
\let\div\relax
\DeclareMathOperator{\div}{div}
\DeclareMathOperator{\vol}{vol}

% Complex Analysis
\let\Re\relax
\DeclareMathOperator{\Re}{Re}
\let\Im\relax
\DeclareMathOperator{\Im}{Im}

% Proofs
\newcommand{\st}{s.t.}
\newcommand{\unique}{!}
\newcommand{\iffdef}{\overset{\ptxt{def}}{\Leftrightarrow}}
\newcommand{\eqdef}{\overset{\ptxt{def}}{=}}
\newcommand{\eqVertical}{\rotatebox[origin=c]{90}{=}}
\newcommand{\mapsfrom}{\mathrel{\reflectbox{\ensuremath{\mapsto}}}}
\newcommand{\mapsdown}{\rotatebox[origin=c]{-90}{$\mapsto$}\mkern2mu}
\newcommand{\mapsup}{\rotatebox[origin=c]{90}{$\mapsto$}\mkern2mu}
\newcommand{\from}{\!\mathrel{\reflectbox{\ensuremath{\to}}}}

% Brackets
\newcommand{\paren}[1]{\left(#1\right)}
\renewcommand{\brack}[1]{\left[#1\right]}
\renewcommand{\brace}[1]{\left\{#1\right\}}
\newcommand{\ang}[1]{\left<#1\right>}

% Algorithms
\algrenewcommand{\algorithmiccomment}[1]{\hskip 1em \texttt{// #1}}
\algrenewcommand\algorithmicrequire{\textbf{Input:}}
\algrenewcommand\algorithmicensure{\textbf{Output:}}
\newcommand{\parSymbol}{\P}
\renewcommand{\P}{\ptxt{\textbf{P}}}
\newcommand{\NP}{\ptxt{\textbf{NP}}}
\newcommand{\NPC}{\ptxt{\textbf{NP-Complete}}}
\newcommand{\NPH}{\ptxt{\textbf{NP-Hard}}}
\newcommand{\EXP}{\ptxt{\textbf{EXP}}}

%%%%%%%%%%%%%%%%%%%%%%%%%%%%%
% Other commands
%%%%%%%%%%%%%%%%%%%%%%%%%%%%%
\newcommand{\flag}[1]{\textbf{\textcolor{red}{#1}}}
\newcommand{\uSym}{\u}
\let\u\relax
\newcommand{\u}[1]{\underline{#1}}
\newcommand{\bSym}{\b}
\let\b\relax
\newcommand{\b}[1]{\textbf{#1}}
\newcommand{\iSym}{\i}
\let\i\relax
\newcommand{\i}[1]{\textit{#1}}

%%%%%%%%%%%%%%%%%%%%%%%%%%%%%
% Make l's curvy in math environments
%%%%%%%%%%%%%%%%%%%%%%%%%%%%%
\mathcode`l="8000
\begingroup
\makeatletter
\lccode`\~=`\l
\DeclareMathSymbol{\lsb@l}{\mathalpha}{letters}{`l}
\lowercase{\gdef~{\ifnum\the\mathgroup=\m@ne \ell \else \lsb@l \fi}}%
\endgroup

\newcommand{\B}{
    \begin{tikzpicture}
    \filldraw [fill=red, draw=black] (0, 0) rectangle (0.37, 0.45);
    \draw [line width=0.5mm, white ] (0.1,0.08) -- (0.1,0.38);
    \draw[line width=0.5mm, white ] (0.1, 0.35) .. controls (0.2, 0.35) and (0.4, 0.2625) .. (0.1, 0.225);
    \draw[line width=0.5mm, white ] (0.1, 0.225) .. controls (0.2, 0.225) and (0.4, 0.1625) .. (0.1, 0.1);
    \end{tikzpicture}
}

\author{Professor David Barrett\\ \small\textit{Transcribed by Thomas Cohn}}
\title{Beginnings of Complex Analysis}
\date{2/20/19} % Can also use \today

\begin{document}
\maketitle
\setlength\RaggedRightParindent{\parindent}
\RaggedRight

\par\noindent Recall, $\begin{array}{r}dz=dx+idy\\ d\conj{z}=dx-idy\end{array}\quad\Rightarrow\quad\begin{array}{l}dx=\frac{dz+d\conj{z}}{2}\\ dy=\frac{dz-d\conj{z}}{2i}\end{array}$.\n

\begin{tikzcd}
	\R^{2} \arrow[r,"f"] \arrow[dr,"\paren{\begin{array}{c}u\\ v\end{array}}", swap] & \C \arrow[d, leftrightarrow]\\
	& \R^{2}
\end{tikzcd}

\par\noindent So we have $\R^{2}\leftrightarrow\C$, with $\paren{\begin{array}{c}x\\ y\end{array}}$ mapping to $x+iy$.\n

\par\noindent When is $fdz$ closed? Well, $fdz=(u+iv)(dx+idy)=(u+iv)dx+(-v+iu)dy$.\n
So we need $D_{1}(-v+iu)=D_{2}(u+iv)$, i.e., $D_{1}u=D_{2}v$ and $D_{2}u=-D_{1}v$.\n
This is equivalent to $\paren{\begin{array}{c}u\\ -v\end{array}}$ is incompressible and irrotational.\n
This is also equivalent to $(u+iv)d\conj{z}$ is closed.\n

\par\noindent From \S{}J, we have
\thm{Given $f\in{}C^{1}(A^{\ptxt{osso}\R^{2}},\C)$, $A$ diffeomorphic to a convex set, then $f(z)dz$ is exact if and only if $f$ satisfies the CR equations.}

\ex{$\displaystyle\frac{dz}{z}=\paren{\frac{x}{x^{2}+y^{2}}-i\frac{y}{x^{2}+y^{2}}}dz$. Check that the CR equations hold.\n
$\displaystyle\frac{dz}{z}=\paren{\frac{x}{x^{2}+y^{2}}dx+\frac{y}{x^{2}+y^{2}}dy}+i\paren{-\frac{y}{x^{2}+y^{2}}dx+\frac{x}{x^{2}+y^{2}}dy}=d\ln(\sqrt{x^{2}+y^{2}})+``id\theta"$.\n
Choose a ray $X$ starting at $0$. Then $\theta_{X}$ is the anti-derivative for the above on $\R^{2}\cut{}X$.\n
We get $\frac{dz}{z}=d(\ln(r)+i\theta_{X})$ on $\R^{2}\cut{}X$.\n
This suggests $\ln_{X}(z)\eqdef\ln(r)+i\theta_{X}$.}

\par\noindent Consider $V,W$ finite-dimensional $\R$-v.s., $f:A^{\ptxt{osso}V}\to{}W$ diffeomorphic at $\vec{x}_{0}\in{}A$.\n
Then we have the affine approximation $\star$ $f(\vec{x})\approx{}f(\vec{x}_{0})+Df(\vec{x}_{0})(\vec{x}-\vec{x}_{0})$.\n

\defn{For $V,W$ $\C$-v.s., $f:V\to{}W$ is $\C$-linear $\iffdef$ $f(\vec{v}_{1}+\vec{v}_{2})=f(\vec{v}_{1})+f(\vec{v}_{2})$ and $f(\lambda\vec{v})=\lambda{}f(\vec{v})$ for $\lambda\in\C$.}

\par\noindent More formally, $Df(\vec{x_{0}})=T\in\Hom_{\R}(V,W)$ means $\lim_{\vec{x}\to\vec{x}_{0}}\frac{f(\vec{x})-f(\vec{x}_{0})-T(\vec{x}-\vec{x}_{0})}{\norm{\vec{x}-\vec{x}_{0}}}=0$. The definition of $\C$-differentiation is exactly the same, except for $T\in\Hom_{\C}(V,W)$. So $f$ is $\C$-differentiable at $\vec{p}$ if and only if $f$ is $\R$-differentiable at $\vec{p}$ and $Df(\vec{p})$ is $\C$-linear.\n

\par\noindent Now, let $V=W=\R^{2}\simeq\C$. Then $Df\paren{\begin{array}{c}u\\ v\end{array}}=\brack{\begin{array}{cc}D_{1}u & D_{2}v\\ D_{1}v & D_{2}v\end{array}}$. Multiplying by $\alpha+i\beta$ is equivalent with multiplying by $\paren{\begin{array}{cc}\alpha & -\beta\\ \beta & \alpha\end{array}}$.\n

\par\noindent $f$ is $\C$-differentiable at $\paren{\begin{array}{c}x_{0}\\ y_{0}\end{array}}$ $\Leftrightarrow$ $f$ is $\R$-differentiable and the CR equations hold.\n

\par\noindent If $f$ is $\C$-differentiable at $\vec{p}$, let $f'_{\C}(z_{0})=(D_{1}u+iD_{1}v)(z_{0})$.\n
So we have the $\C$-affine approximation $f(z)\approx{}f(z_{0})+f'_{\C}(z_{0})(z-z_{0})$.\n

\exer{For $f:A^{\ptxt{osso}\C}\to\C$, we have $f$ $\C$-differentiable at $z_{0}\in{}A$ iff $\lim_{z\to{}z_{0}}\frac{f(z)-f(z_{0})}{z-z_{0}}$ exists (is finite).\n
In this case, we call that limit $f_{\C}'(z_{0})$.}

\thm{For $f=u+iv:A^{\ptxt{osso}\C}\to\C$, the following are equivalent:
\begin{enumerate}[label=\arabic*), topsep=0pt, itemsep=0pt, leftmargin=4\parindent]
	\item $f$ is $C^{1}$ (in the real sense) and $fdz$ is closed.
	\item $u,v\in{}C^{1}(A,\R)$ satisfies the CR equations.
	\item $f$ is $\C$-differentiable at each point of $A$ and $f'_{\C}:A\to\C$ is continuous.
\end{enumerate}}

\defn{If the above holds, we say $f$ is \u{holomorphic} (or \u{analytic}, \u{complex-analytic}) on $A$.}

\section*{Cauchy's Integral Theorem (v.1)}

\par\noindent Given $A^{\ptxt{osso}\C}$ diffeomorphic to a convex set, $f$ holomorphic on $A$, $\alpha:[a,b]\to{}A$ piecewise $C^{1}$, and\n
$\alpha(a)=\alpha(b)$, then $\int_{Y_{\alpha}}f\,dz=0$.\n
Proof: Hyp $\overset{J.8}{\Rightarrow}fdz$ exact $\overset{J.1}{\Rightarrow}\int_{Y_{\alpha}}f\,dz=0$.\proven

\ex{Some holomorphic functions:
\begin{enumerate}[label=(\arabic*), topsep=0pt, itemsep=0pt]
	\item $f(z)=C$ (for constant $C$).
	\item $f(z)=z=x+iy$.
	\item $f(z)=z\inv=\frac{x}{x^{2}+y^{2}}-i\frac{y}{x^{2}+y^{2}}$.
	\item $f(z)=\log_{X}(z)=\log\abs{z}+i\theta_{X}(z)$. Note that the inverse of $\log_{X}$ is $\exp:x+iy\mapsto\cos{}y+ie^{x}\sin{}y$, and does not depend on $X$.
	\item $f(z)=e^{z}$ (verify using method 1 or method 2).
\end{enumerate}}

\ex{Some not holomorphic functions:
\begin{enumerate}[label=(\arabic*), topsep=0pt, itemsep=0pt]
	\item $f(z)=\conj{z}$.
	\item $f(z)=\Re(z)$.
	\item $f(z)=\Im(z)$.
\end{enumerate}}

\thm{If $f,g$ holomorphic, then the following are holomorphic wherever defined:
\begin{enumerate}[label=(\alph*), topsep=0pt, itemsep=0pt]
	\item $f+g$
	\item $f-g$
	\item $fg$
	\item $f/g$
	\item $f\of{}g$
\end{enumerate}\up\n
Proof: Method 1}

\par\noindent Alternative methods: (a)(b) using CR equations, and (e) using the fact that the $\R$-affine approximation for $f\of{}g$ is the composition of the corresponding $\R$-affine approximations. If these are, in fact, $\C$-affine, then so is the first one.\n

\cor{From (e), $g$ holomorphic $\Rightarrow$ $1/g$ holomorphic where defined.}

\lemma{$f(z)=z^{2}=(x^{2}-y^{2})+i2xy$ is holomorphic.\n
Proof: Use CR equations.}

\cor{$f,g$ holomorphic $\Rightarrow$ $fg=\frac{(f+g)^{2}-(f-g)^{2}}{2}$ holomorphic.}

\cor{$f,g$ holomorphic $\Rightarrow$ $f/g$ holomorphic where defined.}

\end{document}