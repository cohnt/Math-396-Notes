\documentclass[10pt,letterpaper]{article}
\usepackage[utf8]{inputenc}
\usepackage[intlimits]{amsmath}
\usepackage{amsfonts}
\usepackage{amssymb}
\usepackage{ragged2e}
\usepackage[letterpaper, margin=1in]{geometry}
\usepackage{graphicx}
\usepackage{cancel}
\usepackage{mathtools}
\usepackage{tabularx}
\usepackage{arydshln}
\usepackage{tensor}
\usepackage{array}
\usepackage{xcolor}
\usepackage[boxed]{algorithm}
\usepackage[noend]{algpseudocode}
\usepackage{listings}
\usepackage{textcomp}
\usepackage[pdf,tmpdir,singlefile]{graphviz}
\usepackage{mathrsfs}
\usepackage{bbm}
\usepackage{tikz}
\usepackage{enumitem}
\usepackage{arydshln}
\usepackage{relsize}

%%%%%%%%%%%%%%%%%%%%%%%%%%%%%
% Formatting commands
%%%%%%%%%%%%%%%%%%%%%%%%%%%%%
\newcommand{\n}{\hfill\break}
\newcommand{\up}{\vspace{-\baselineskip}}
\newcommand{\lemma}[1]{\par\noindent\settowidth{\hangindent}{\textbf{Lemma: }}\textbf{Lemma: }#1}
\newcommand{\defn}[1]{\par\noindent\settowidth{\hangindent}{\textbf{Defn: }}\textbf{Defn: }#1\n}
\newcommand{\thm}[1]{\par\noindent\settowidth{\hangindent}{\textbf{Thm: }}\textbf{Thm: }#1\n}
\newcommand{\prop}[1]{\par\noindent\settowidth{\hangindent}{\textbf{Prop: }}\textbf{Prop: }#1\n}
\newcommand{\cor}[1]{\par\noindent\settowidth{\hangindent}{\textbf{Cor: }}\textbf{Cor: }#1\n}
\newcommand{\ex}[1]{\par\noindent\settowidth{\hangindent}{\textbf{Ex: }}\textbf{Ex: }#1\n}
\newcommand{\exer}[1]{\par\noindent\settowidth{\hangindent}{\textbf{Exer: }}\textbf{Exer: }#1\n}
\newcommand{\proven}{\;$\square$\n}
\newcommand{\problem}[1]{\par\noindent{#1}\n}
\newcommand{\problempart}[2]{\par\noindent\indent{}\settowidth{\hangindent}{\textbf{(#1)} \indent{}}\textbf{(#1)} #2\n}
\newcommand{\ptxt}[1]{\textrm{\textnormal{#1}}}
\newcommand{\inlineeq}[1]{\centerline{$\displaystyle #1$}}
\newcommand{\pageline}{\noindent\rule{\textwidth}{0.1pt}}

%%%%%%%%%%%%%%%%%%%%%%%%%%%%%
% Math commands
%%%%%%%%%%%%%%%%%%%%%%%%%%%%%
% Set Theory
\newcommand{\card}[1]{\left|#1\right|}
\newcommand{\set}[1]{\left\{#1\right\}}
\newcommand{\setmid}{\;\middle|\;}
\newcommand{\ps}[1]{\mathcal{P}\left(#1\right)}
\newcommand{\pfinite}[1]{\mathcal{P}^{\ptxt{finite}}\left(#1\right)}
\newcommand{\naturals}{\mathbb{N}}
\newcommand{\N}{\naturals}
\newcommand{\integers}{\mathbb{Z}}
\newcommand{\Z}{\integers}
\newcommand{\rationals}{\mathbb{Q}}
\newcommand{\Q}{\rationals}
\newcommand{\reals}{\mathbb{R}}
\newcommand{\R}{\reals}
\newcommand{\complex}{\mathbb{C}}
\newcommand{\C}{\complex}
\newcommand{\halfPlane}{\mathbb{H}}
\let\H\relax
\newcommand{\H}{\halfPlane}
\newcommand{\comp}{^{\complement}}
\DeclareMathOperator{\Hom}{Hom}
\newcommand{\Ind}{\mathbbm{1}}
\newcommand{\cut}{\setminus}

% Graph Theory
\let\deg\relax
\DeclareMathOperator{\deg}{deg}
\newcommand{\degp}{\ptxt{deg}^{+}}
\newcommand{\degn}{\ptxt{deg}^{-}}
\newcommand{\precdot}{\mathrel{\ooalign{$\prec$\cr\hidewidth\hbox{$\cdot\mkern0.5mu$}\cr}}}
\newcommand{\succdot}{\mathrel{\ooalign{$\cdot\mkern0.5mu$\cr\hidewidth\hbox{$\succ$}\cr\phantom{$\succ$}}}}
\DeclareMathOperator{\cl}{cl}
\DeclareMathOperator{\affdim}{affdim}

% Probability
\newcommand{\Prob}{\mathbb{P}}
\newcommand{\Avg}{\mathbb{E}}

% Standard Math
\newcommand{\inv}{^{-1}}
\newcommand{\abs}[1]{\left|#1\right|}
\newcommand{\ceil}[1]{\left\lceil{}#1\right\rceil{}}
\newcommand{\floor}[1]{\left\lfloor{}#1\right\rfloor{}}
\newcommand{\conj}[1]{\overline{#1}}
\newcommand{\of}{\circ}
\newcommand{\tri}{\triangle}
\newcommand{\inj}{\hookrightarrow}
\newcommand{\surj}{\twoheadrightarrow}
\newcommand{\mapsfrom}{\mathrel{\reflectbox{\ensuremath{\mapsto}}}}
\newcommand{\mapsdown}{\rotatebox[origin=c]{-90}{$\mapsto$}\mkern2mu}
\newcommand{\mapsup}{\rotatebox[origin=c]{90}{$\mapsto$}\mkern2mu}
\newcommand{\ndiv}{\nmid}
\renewcommand{\epsilon}{\varepsilon}
\newcommand{\divides}{\mid}
\newcommand{\ndivides}{\nmid}
\DeclareMathOperator{\lcm}{lcm}
\newcommand{\eqVertical}{\rotatebox[origin=c]{90}{=}}

% Linear Algebra
\newcommand{\Id}{\textrm{\textnormal{Id}}}
\newcommand{\im}{\textrm{\textnormal{im}}}
\newcommand{\norm}[1]{\abs{\abs{#1}}}
\newcommand{\tpose}{^{T}}
\newcommand{\iprod}[1]{\left<#1\right>}
\DeclareMathOperator{\trace}{tr}
\newcommand{\chgBasMat}[3]{\!\!\tensor*[_{#1}]{\left[#2\right]}{_{#3}}}
\newcommand{\vecBas}[2]{\tensor*[]{\left[#1\right]}{_{#2}}}
\DeclareMathOperator{\GL}{GL}
\DeclareMathOperator{\Mat}{Mat}
\DeclareMathOperator{\vspan}{span}
\DeclareMathOperator{\rank}{rank}
\newcommand{\V}[1]{\vec{#1}}

% Topology
\newcommand{\closure}[1]{\overline{#1}}
\newcommand{\uball}{\mathcal{U}}
\DeclareMathOperator{\Int}{Int}
\DeclareMathOperator{\Ext}{Ext}
\DeclareMathOperator{\Bd}{Bd}
\DeclareMathOperator{\rInt}{rInt}
\DeclareMathOperator{\ch}{ch}
\DeclareMathOperator{\ah}{ah}
\newcommand{\Tau}{\mathlarger{\mathlarger{\mathlarger{\mathlarger{\tau}}}}}

% Analysis
\DeclareMathOperator{\Graph}{Graph}
\DeclareMathOperator{\epi}{epi}
\DeclareMathOperator{\hypo}{hypo}
\DeclareMathOperator{\supp}{supp}
\newcommand{\lint}[2]{\underset{#1}{\overset{#2}{{\color{black}\underline{{\color{white}\overline{{\color{black}\int}}\color{black}}}}}}}
\newcommand{\uint}[2]{\underset{#1}{\overset{#2}{{\color{white}\underline{{\color{black}\overline{{\color{black}\int}}\color{black}}}}}}}
\newcommand{\alignint}[2]{\underset{#1}{\overset{#2}{{\color{white}\underline{{\color{white}\overline{{\color{black}\int}}\color{black}}}}}}}
\newcommand{\extint}{\ptxt{ext}\int}
\newcommand{\extalignint}[2]{\ptxt{ext}\alignint{#1}{#2}}
\newcommand{\conv}{\ast}

% Proofs
\newcommand{\st}{s.t.}
\newcommand{\unique}{!}
\newcommand{\iffdef}{\overset{\ptxt{def}}{\Leftrightarrow}}
\newcommand{\eqdef}{\overset{\ptxt{def}}{=}}

% Brackets
\newcommand{\paren}[1]{\left(#1\right)}
\renewcommand{\brack}[1]{\left[#1\right]}
\renewcommand{\brace}[1]{\left\{#1\right\}}
\newcommand{\ang}[1]{\left<#1\right>}

% Algorithms
\algrenewcommand{\algorithmiccomment}[1]{\hskip 1em \texttt{// #1}}
\algrenewcommand\algorithmicrequire{\textbf{Input:}}
\algrenewcommand\algorithmicensure{\textbf{Output:}}
\newcommand{\parSymbol}{\P}
\renewcommand{\P}{\ptxt{\textbf{P}}}
\newcommand{\NP}{\ptxt{\textbf{NP}}}
\newcommand{\NPC}{\ptxt{\textbf{NP-Complete}}}
\newcommand{\NPH}{\ptxt{\textbf{NP-Hard}}}
\newcommand{\EXP}{\ptxt{\textbf{EXP}}}

%%%%%%%%%%%%%%%%%%%%%%%%%%%%%
% Other commands
%%%%%%%%%%%%%%%%%%%%%%%%%%%%%
\newcommand{\flag}[1]{\textbf{\textcolor{red}{#1}}}

%%%%%%%%%%%%%%%%%%%%%%%%%%%%%
% Make l's curvy in math environments
%%%%%%%%%%%%%%%%%%%%%%%%%%%%%
\mathcode`l="8000
\begingroup
\makeatletter
\lccode`\~=`\l
\DeclareMathSymbol{\lsb@l}{\mathalpha}{letters}{`l}
\lowercase{\gdef~{\ifnum\the\mathgroup=\m@ne \ell \else \lsb@l \fi}}%
\endgroup

\newcommand{\B}{
    \begin{tikzpicture}
    \filldraw [fill=red, draw=black] (0, 0) rectangle (0.37, 0.45);
    \draw [line width=0.5mm, white ] (0.1,0.08) -- (0.1,0.38);
    \draw[line width=0.5mm, white ] (0.1, 0.35) .. controls (0.2, 0.35) and (0.4, 0.2625) .. (0.1, 0.225);
    \draw[line width=0.5mm, white ] (0.1, 0.225) .. controls (0.2, 0.225) and (0.4, 0.1625) .. (0.1, 0.1);
    \end{tikzpicture}
}

\author{Professor David Barrett\\ \small\textit{Transcribed by Thomas Cohn}}
\title{Integral Manifolds}
\date{1/11/19} % Can also use \today

\begin{document}
\maketitle
\setlength\RaggedRightParindent{\parindent}
\RaggedRight

\par\noindent Throwback to 11/21/18...\n

\par\noindent Recall $\int_{Y_{\alpha}}f\,dV$, where $Y_{\alpha}$ is a parameterized $k$-manifold.\n
We also want to know what $\int_{M}f\,dV$, where $M$ is a $k$-manifold.\n

\par\noindent For now, we will focus on the case where $M$ is compact and $f$ is continuous.\n

\par\noindent Special case: Assume $\supp{}f\subset{}V$ with $\alpha:U\to{}V\subset{}M$ coordinate patch.\n
Then define $\int_{M}f\,dV=\int_{V_{\alpha}}f\,dV=\int_{U}(f\of\alpha)V(D\alpha)$. This is guaranteed to exist ``in the ordinary sense''.\n

\prop{This does not depend on our choice of coordinate patch.\n
Proof: Suppose we also have $\tilde{\alpha}:\tilde{U}\to\tilde{V}\subset{}M$. We can replace $V$ and $\tilde{V}$ with $V\cap\tilde{V}$, so we may assume $V=\tilde{V}$. $\tilde{\alpha}=\alpha\of(\alpha\inv\of\tilde{\alpha})$, and $\alpha\inv\of\tilde{\alpha}$ is a transition map, so from a result we proved on 11/21/18, $\int_{V_{\alpha}}f\,dV=\int_{V_{\tilde{\alpha}}}f\,dV$.\proven}

\up
\par\noindent\textit{But}, what if we require multiple cooridnate patches to cover $\supp{}f$?\n

\par\noindent Choose coordinate patches $\alpha_{j}:U_{j}\to{}V_{j}\subset{}M$ for $j\in\set{1,2,\ldots,N}$, with $M=V_{1}\cup\cdots\cup{}V_{N}$ (we can assume there are a finite number of $V_{i}$ because $M$ is compact). Write $V_{j}=M\cap{}E_{j}$ with $E_{j}^{\ptxt{open}}\subseteq\R^{n}$.\n

\par\noindent We can write $1=\varphi_{1}+\cdots+\varphi_{N}$ on $E_{1}\cup\cdots\cup{}E_{N}$ with $\supp\varphi_{j}\subset{}E_{j}$ and $(\supp\varphi_{j})\cap{}M\subset{}V_{j}$.\n
So $f=f\varphi_{1}+\cdots+f\varphi_{N}$. Thus, we define\n

\defn{$\displaystyle\int_{M}f\,dV=\int_{M}f\varphi_{1}\,dV+\cdots+\int_{M}f\varphi_{N}\,dV$}

\par\noindent Of course, we need to check that we get the same result using $1=\tilde{\varphi_{1}}+\cdots+\tilde{\varphi_{n}}$.

\[
\begin{array}{ccccc}
\sum_{j}\int{}f\varphi_{j}\,dV & \overset{\ptxt{?}}{=} & \sum_{k}\int{}f\tilde{\varphi_{k}}\,dV\\
\eqVertical & & \eqVertical\\
\sum_{j}\int{}f\varphi_{j}\paren{\sum_{k}\tilde{\varphi_{k}}}\,dV & & \sum_{k}\int{}f\tilde{\varphi_{k}}\paren{\sum_{j}\varphi_{j}}\,dV\\
\eqVertical & & \eqVertical\\
\sum_{j}\sum_{k}\int{}f\varphi_{j}\tilde{\varphi_{k}}\,dV & = & \sum_{j}\sum_{k}\int{}f\varphi_{j}\tilde{\varphi_{k}}\,dV
\end{array}
\]

\newpage

\par\noindent\textbf{Integral Manifolds}\n

\par\noindent Consider $\omega$, a $1$-form on $A^{\ptxt{open}}\subseteq\R^{n}$.\n
Then $\omega:A\to\paren{\R^{n}}^{*}=\Hom(\R^{n},\R)$, and $\omega(\vec{p}):\R^{n}\to\R$ is a linear map.\n

\par\noindent Usually, $\dim(\ker(\omega(\vec{p})))=n-1$, but sometimes it's $n$. Consider $\vec{p}+\ker(\omega(\vec{p}))$, an affine set.\n

\exer{(HW 1 \#3) Prove for a $k$-manifold $M\subset{}A$ that the following are equivalent:
\begin{enumerate}[label=(\alph*)]
	\item $\Tau_{p}M\subset\ker(\omega(\vec{p}))$, $\forall\vec{p}\in{}M$
	\item $\alpha^{*}\omega=0$, $\forall\alpha$ coordinate patch for $M$
	\item $\int_{C}\omega=0$, $\forall{}C^{1\ptxt{-mfd}}\subset{}M$
\end{enumerate}}

\up\defn{If $M$ satisfies these conditions, we say that $M$ is an \underline{integral manifold} for $\omega$.}

\end{document}