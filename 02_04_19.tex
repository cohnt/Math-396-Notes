\documentclass[10pt,letterpaper]{article}
\usepackage[utf8]{inputenc}
\usepackage[intlimits]{amsmath}
\usepackage{amsfonts}
\usepackage{amssymb}
\usepackage{ragged2e}
\usepackage[letterpaper, margin=1in]{geometry}
\usepackage{graphicx}
\usepackage{cancel}
\usepackage{mathtools}
\usepackage{tabularx}
\usepackage{arydshln}
\usepackage{tensor}
\usepackage{array}
\usepackage{xcolor}
\usepackage[boxed]{algorithm}
\usepackage[noend]{algpseudocode}
\usepackage{listings}
\usepackage{textcomp}
\usepackage[pdf,tmpdir,singlefile]{graphviz}
\usepackage{mathrsfs}
\usepackage{bbm}
\usepackage{tikz}
\usepackage{enumitem}
\usepackage{arydshln}
\usepackage{relsize}
\usepackage{multicol}

%%%%%%%%%%%%%%%%%%%%%%%%%%%%%
% Formatting commands
%%%%%%%%%%%%%%%%%%%%%%%%%%%%%
\newcommand{\n}{\hfill\break}
\newcommand{\up}{\vspace{-\baselineskip}}
\newcommand{\lemma}[1]{\par\noindent\settowidth{\hangindent}{\textbf{Lemma: }}\textbf{Lemma: }#1}
\newcommand{\defn}[1]{\par\noindent\settowidth{\hangindent}{\textbf{Defn: }}\textbf{Defn: }#1\n}
\newcommand{\thm}[1]{\par\noindent\settowidth{\hangindent}{\textbf{Thm: }}\textbf{Thm: }#1\n}
\newcommand{\prop}[1]{\par\noindent\settowidth{\hangindent}{\textbf{Prop: }}\textbf{Prop: }#1\n}
\newcommand{\cor}[1]{\par\noindent\settowidth{\hangindent}{\textbf{Cor: }}\textbf{Cor: }#1\n}
\newcommand{\ex}[1]{\par\noindent\settowidth{\hangindent}{\textbf{Ex: }}\textbf{Ex: }#1\n}
\newcommand{\exer}[1]{\par\noindent\settowidth{\hangindent}{\textbf{Exer: }}\textbf{Exer: }#1\n}
\newcommand{\proven}{\;$\square$\n}
\newcommand{\problem}[1]{\par\noindent{#1}\n}
\newcommand{\problempart}[2]{\par\noindent\indent{}\settowidth{\hangindent}{\textbf{(#1)} \indent{}}\textbf{(#1)} #2\n}
\newcommand{\ptxt}[1]{\textrm{\textnormal{#1}}}
\newcommand{\inlineeq}[1]{\centerline{$\displaystyle #1$}}
\newcommand{\pageline}{\noindent\rule{\textwidth}{0.1pt}}

%%%%%%%%%%%%%%%%%%%%%%%%%%%%%
% Math commands
%%%%%%%%%%%%%%%%%%%%%%%%%%%%%
% Set Theory
\newcommand{\card}[1]{\left|#1\right|}
\newcommand{\set}[1]{\left\{#1\right\}}
\newcommand{\setmid}{\;\middle|\;}
\newcommand{\ps}[1]{\mathcal{P}\left(#1\right)}
\newcommand{\pfinite}[1]{\mathcal{P}^{\ptxt{finite}}\left(#1\right)}
\newcommand{\naturals}{\mathbb{N}}
\newcommand{\N}{\naturals}
\newcommand{\integers}{\mathbb{Z}}
\newcommand{\Z}{\integers}
\newcommand{\rationals}{\mathbb{Q}}
\newcommand{\Q}{\rationals}
\newcommand{\reals}{\mathbb{R}}
\newcommand{\R}{\reals}
\newcommand{\complex}{\mathbb{C}}
\newcommand{\C}{\complex}
\newcommand{\halfPlane}{\mathbb{H}}
\let\H\relax
\newcommand{\H}{\halfPlane}
\newcommand{\comp}{^{\complement}}
\DeclareMathOperator{\Hom}{Hom}
\newcommand{\Ind}{\mathbbm{1}}
\newcommand{\cut}{\setminus}
\DeclareMathOperator{\elem}{elem}

% Graph Theory
\let\deg\relax
\DeclareMathOperator{\deg}{deg}
\newcommand{\degp}{\ptxt{deg}^{+}}
\newcommand{\degn}{\ptxt{deg}^{-}}
\newcommand{\precdot}{\mathrel{\ooalign{$\prec$\cr\hidewidth\hbox{$\cdot\mkern0.5mu$}\cr}}}
\newcommand{\succdot}{\mathrel{\ooalign{$\cdot\mkern0.5mu$\cr\hidewidth\hbox{$\succ$}\cr\phantom{$\succ$}}}}
\DeclareMathOperator{\cl}{cl}
\DeclareMathOperator{\affdim}{affdim}

% Probability
\newcommand{\Prob}{\mathbb{P}}
\newcommand{\Avg}{\mathbb{E}}

% Standard Math
\newcommand{\inv}{^{-1}}
\newcommand{\abs}[1]{\left|#1\right|}
\newcommand{\ceil}[1]{\left\lceil{}#1\right\rceil{}}
\newcommand{\floor}[1]{\left\lfloor{}#1\right\rfloor{}}
\newcommand{\conj}[1]{\overline{#1}}
\newcommand{\of}{\circ}
\newcommand{\tri}{\triangle}
\newcommand{\inj}{\hookrightarrow}
\newcommand{\surj}{\twoheadrightarrow}
\newcommand{\ndiv}{\nmid}
\renewcommand{\epsilon}{\varepsilon}
\newcommand{\divides}{\mid}
\newcommand{\ndivides}{\nmid}
\DeclareMathOperator{\lcm}{lcm}
\DeclareMathOperator{\sgn}{sgn}
\newcommand{\map}[4]{\!\!\!\begin{array}[t]{rcl}#1 & \!\!\!\!\to & \!\!\!\!#2\\ #3 & \!\!\!\!\mapsto & \!\!\!\!#4\end{array}}
\newcommand{\bigsum}[2]{\smashoperator[lr]{\sum_{\scalebox{#1}{$#2$}}}}

% Linear Algebra
\newcommand{\Id}{\textrm{\textnormal{Id}}}
\newcommand{\im}{\textrm{\textnormal{im}}}
\newcommand{\norm}[1]{\abs{\abs{#1}}}
\newcommand{\tpose}{^{T}}
\newcommand{\iprod}[1]{\left<#1\right>}
\DeclareMathOperator{\trace}{tr}
\newcommand{\chgBasMat}[3]{\!\!\tensor*[_{#1}]{\left[#2\right]}{_{#3}}}
\newcommand{\vecBas}[2]{\tensor*[]{\left[#1\right]}{_{#2}}}
\DeclareMathOperator{\GL}{GL}
\DeclareMathOperator{\Mat}{Mat}
\DeclareMathOperator{\vspan}{span}
\DeclareMathOperator{\rank}{rank}
\newcommand{\V}[1]{\vec{#1}}
\DeclareMathOperator{\proj}{proj}
\DeclareMathOperator{\compProj}{comp}

% Multilinear Algebra
\newcommand{\Lsym}{\L}
\let\L\relax
\DeclareMathOperator{\L}{\mathscr{L}}
\DeclareMathOperator{\A}{\mathcal{A}}
\DeclareMathOperator{\Alt}{Alt}
\DeclareMathOperator{\Sym}{Sym}
\newcommand{\ot}{\otimes}
\newcommand{\ox}{\otimes}
\DeclareMathOperator{\asc}{asc}
\DeclareMathOperator{\asSet}{set}
\DeclareMathOperator{\sort}{sort}
\DeclareMathOperator{\ringA}{\mathring{A}}

% Topology
\newcommand{\closure}[1]{\overline{#1}}
\newcommand{\uball}{\mathcal{U}}
\DeclareMathOperator{\Int}{Int}
\DeclareMathOperator{\Ext}{Ext}
\DeclareMathOperator{\Bd}{Bd}
\DeclareMathOperator{\rInt}{rInt}
\DeclareMathOperator{\ch}{ch}
\DeclareMathOperator{\ah}{ah}
\newcommand{\Tau}{\mathlarger{\mathlarger{\mathlarger{\mathlarger{\tau}}}}}

% Analysis
\DeclareMathOperator{\Graph}{Graph}
\DeclareMathOperator{\epi}{epi}
\DeclareMathOperator{\hypo}{hypo}
\DeclareMathOperator{\supp}{supp}
\newcommand{\lint}[2]{\underset{#1}{\overset{#2}{{\color{black}\underline{{\color{white}\overline{{\color{black}\int}}\color{black}}}}}}}
\newcommand{\uint}[2]{\underset{#1}{\overset{#2}{{\color{white}\underline{{\color{black}\overline{{\color{black}\int}}\color{black}}}}}}}
\newcommand{\alignint}[2]{\underset{#1}{\overset{#2}{{\color{white}\underline{{\color{white}\overline{{\color{black}\int}}\color{black}}}}}}}
\newcommand{\extint}{\ptxt{ext}\int}
\newcommand{\extalignint}[2]{\ptxt{ext}\alignint{#1}{#2}}
\newcommand{\conv}{\ast}

% Proofs
\newcommand{\st}{s.t.}
\newcommand{\unique}{!}
\newcommand{\iffdef}{\overset{\ptxt{def}}{\Leftrightarrow}}
\newcommand{\eqdef}{\overset{\ptxt{def}}{=}}
\newcommand{\eqVertical}{\rotatebox[origin=c]{90}{=}}
\newcommand{\mapsfrom}{\mathrel{\reflectbox{\ensuremath{\mapsto}}}}
\newcommand{\mapsdown}{\rotatebox[origin=c]{-90}{$\mapsto$}\mkern2mu}
\newcommand{\mapsup}{\rotatebox[origin=c]{90}{$\mapsto$}\mkern2mu}

% Brackets
\newcommand{\paren}[1]{\left(#1\right)}
\renewcommand{\brack}[1]{\left[#1\right]}
\renewcommand{\brace}[1]{\left\{#1\right\}}
\newcommand{\ang}[1]{\left<#1\right>}

% Algorithms
\algrenewcommand{\algorithmiccomment}[1]{\hskip 1em \texttt{// #1}}
\algrenewcommand\algorithmicrequire{\textbf{Input:}}
\algrenewcommand\algorithmicensure{\textbf{Output:}}
\newcommand{\parSymbol}{\P}
\renewcommand{\P}{\ptxt{\textbf{P}}}
\newcommand{\NP}{\ptxt{\textbf{NP}}}
\newcommand{\NPC}{\ptxt{\textbf{NP-Complete}}}
\newcommand{\NPH}{\ptxt{\textbf{NP-Hard}}}
\newcommand{\EXP}{\ptxt{\textbf{EXP}}}

%%%%%%%%%%%%%%%%%%%%%%%%%%%%%
% Other commands
%%%%%%%%%%%%%%%%%%%%%%%%%%%%%
\newcommand{\flag}[1]{\textbf{\textcolor{red}{#1}}}
\newcommand{\uSym}{\u}
\let\u\relax
\newcommand{\u}[1]{\underline{#1}}
\newcommand{\bSym}{\b}
\let\b\relax
\newcommand{\b}[1]{\textbf{#1}}
\newcommand{\iSym}{\i}
\let\i\relax
\newcommand{\i}[1]{\textit{#1}}

%%%%%%%%%%%%%%%%%%%%%%%%%%%%%
% Make l's curvy in math environments
%%%%%%%%%%%%%%%%%%%%%%%%%%%%%
\mathcode`l="8000
\begingroup
\makeatletter
\lccode`\~=`\l
\DeclareMathSymbol{\lsb@l}{\mathalpha}{letters}{`l}
\lowercase{\gdef~{\ifnum\the\mathgroup=\m@ne \ell \else \lsb@l \fi}}%
\endgroup

\newcommand{\B}{
    \begin{tikzpicture}
    \filldraw [fill=red, draw=black] (0, 0) rectangle (0.37, 0.45);
    \draw [line width=0.5mm, white ] (0.1,0.08) -- (0.1,0.38);
    \draw[line width=0.5mm, white ] (0.1, 0.35) .. controls (0.2, 0.35) and (0.4, 0.2625) .. (0.1, 0.225);
    \draw[line width=0.5mm, white ] (0.1, 0.225) .. controls (0.2, 0.225) and (0.4, 0.1625) .. (0.1, 0.1);
    \end{tikzpicture}
}

\author{Professor David Barrett\\ \small\textit{Transcribed by Thomas Cohn}}
\title{Integrating $1$-Forms}
\date{2/4/19} % Can also use \today

\begin{document}
\maketitle
\setlength\RaggedRightParindent{\parindent}
\RaggedRight

\par\noindent In $\R^{2}$, $d(\alpha\,dx+\beta\,dy)=\paren{\frac{\partial\beta}{\partial{}x}-\frac{\partial\alpha}{\partial{}y}}\,dx\wedge{}dy$.\n

\thm{Green's Thm (Rectangle Version -- Lemma J.7)\n
\inlineeq{
	\int_{\partial{}R^{\ptxt{box}}}(\alpha\,dx+\beta\,dy)=\int_{R}\paren{\frac{\partial\beta}{\partial{}x}-\frac{\partial\alpha}{\partial{}y}}
}}

\par\noindent Alternatively, $\displaystyle\int_{\partial}\omega=\int_{R}\paren{\frac{\partial\beta}{\partial{}x}-\frac{\partial\alpha}{\partial{}y}}\,dx\wedge{}dy=\int_{R}d\omega$.\n

\defn{For $C^{r}$ $k$-form $\displaystyle\omega=\sum_{\substack{I\asc\\ k\ptxt{-tuple}}}b_{I}(\vec{x})\,dx_{i_{1}}\wedge\cdots\wedge{}dx_{i_{k}}$, $\displaystyle{}d\omega\eqdef\sum_{\substack{I\asc\\ k\ptxt{-tuple}}}db_{I}\wedge{}dx_{i_{1}}\wedge\cdots\wedge{}dx_{i_{k}}$. $d\omega$ is at least $C^{r-1}$.}

\section*{$1$-forms on $\R^{n}$}

\par\noindent We still have $\omega$ closed $\Leftrightarrow$ $d\omega=0$.\n

\prop{(15) $d(\omega_{1}+\omega_{2})=d\omega_{1}+d\omega_{2}$ (for $\deg(\omega_{1})=\deg(\omega_{2})$).}

\prop{$d(\omega_{1}\wedge\omega_{2})=d\omega_{1}\wedge\omega_{2}+(-1)^{\deg\omega_{1}}\omega_{1}\wedge{}d\omega_{2}$ (note that $\deg(\omega_{1})$ is not necessarily equal to $\deg(\omega_{2})$).\n
Proof: Note that $d(fg)=f\,dg+g\,df$ for scalar functions $f$ and $g$ (395 rule (14)).\n
$d(\omega_{1}\wedge\omega_{2})$ is gross, so we'll go term by term.\n
\inlineeq{
	\begin{array}{rcl}
		d(\alpha_{I}\Psi_{I}\wedge\beta_{J}\Psi_{J}) & \!\!\!\!= & \!\!\!\!d(\alpha_{I}\beta_{J})\wedge\Psi_{I}\wedge\Psi_{J}\\
		& \!\!\!\!= & \!\!\!\!d(\alpha_{I}\beta_{J})\wedge\Psi_{I}\wedge\Psi_{J}\\
		& \!\!\!\!= & \!\!\!\!d\alpha_{I}\beta_{J}\wedge\Psi_{I}\wedge\Psi_{J}+\alpha{I}\,d\beta_{J}\wedge\Psi_{I}\wedge\Psi_{J}\\
		& \!\!\!\!= & \!\!\!\!d(\alpha_{I}\Psi_{I})\wedge(\beta_{J}\Psi_{J})+(-1)^{\deg\omega}(\alpha_{I}\Psi_{I})\,d(\beta_{J}\Psi_{J})
	\end{array}
}
\proven}

\prop{(17) $dd\omega=0$ (assuming $\omega$ is $C^{2}$).\n
Proof: If $\deg\omega=0$, then we're done because exact $1$-forms are closed.\n
In general, $dd\paren{\sum{}\alpha_{I}\Psi_{I}}=d\paren{\sum{}d\alpha_{I}\wedge{}dx_{I}}=\cancel{\sum{}dd\alpha_{I}\wedge{}dx_{I}}\pm\cancel{\sum{}d\alpha_{I}\wedge{}dd{}x_{I}}=0$.\proven}

\prop{$d(\Phi^{*}\omega)=\Phi^{*}d\omega$.}

\par\noindent Proof of prop: We already know this to be true for $\deg\omega=0$.\n
In general:
\begin{align*}
	d(\Phi^{*}\omega) & =d\paren{\Phi^{*}\paren{\sum{}b_{I}dx_{i_{1}}\wedge\cdots\wedge{}dx_{i_{k}}}}\\
	& =\sum{}d\paren{\Phi^{*}b_{I}\cdot\Phi^{*}(dx_{i_{1}})\wedge\cdots\wedge\Phi^{*}(dx_{i_{k}})}\\
	& =\sum{}d\paren{\Phi^{*}b_{I}\cdot{}d(\Phi^{*}x_{i_{1}})\wedge\cdots\wedge{}d(\Phi^{*}(x_{i_{k}}))}\\
	& =\sum{}\Phi^{*}(db_{I})\cdot\Phi^{*}(dx_{i_{1}})\wedge\cdots\wedge\Phi^{*}(dx_{i_{k}})\\
	& =\Phi^{*}\paren{\sum{}db_{I}dx_{I}}\\
	& =\Phi^{*}d\omega
\end{align*}\up\n
\proven

\section*{Integration}

\subsection{Integrating $k$-forms over Open Subsets of $\R^{k}$}

\par\noindent Let $U^{\ptxt{open}}\subset\R^{k}$ (or $\H^{k}$).

\defn{$\omega=f\,dx_{1}\wedge\cdots\wedge{}dx_{k}$. $\int\omega\eqdef\int_{U}f$.}

\par\noindent Existence is guaranteed if $\supp{}f$ is compact (because then we can cover $\supp{}f$ with finitely many closed boxes contained in $U$).\n

\par\noindent Consider $\Phi^{\ptxt{diffeo}}U^{\ptxt{osso}\R^{k}\ptxt{ or }\H^{k}}\to{}V^{\ptxt{osso}\R^{k}\ptxt{ or }\H^{k}}$, $\omega=f\,dx_{1}\wedge\cdots\wedge{}dx_{k}$ $k$-form on $V$. Then
\begin{align*}
	\int_{U}\Phi^{*}\omega & =\int_{U}(\Phi^{*}f)\Phi^{*}dx_{1}\wedge\cdots\wedge\Phi^{*}dx_{k}\\
	& =\int_{U}(\Phi^{*}f)\,d(\Phi_{1})\wedge\cdots\wedge{}d(\Phi_{k})\\
	& =\int_{U}f\of\Phi{}h(D\Phi)\,dx_{1}\wedge\cdots\wedge{}dx_{k}\\
	& =I\int_{V}f
\end{align*}

\par\noindent Note that $h(D\Phi)$ is an alternating multilinear function of the rows of $D\Phi$. $h(I)=1$, and $h(D\Phi)=\det{}D\Phi$. So $\Phi^{*}(dx_{1}\wedge\cdots\wedge{}dx_{k})=(\det{}D\Phi)dx_{1}\wedge\cdots\wedge{}dx_{k}$.\n

\par\noindent Also, $I$ is positive if $\det{}D\Phi>0$, and $-$ if $\det{}D\Phi<0$. Split it into integrals on the connected components if $U$ is disconnected.\n

\newpage

\subsection{Integrating $k$-forms over Parameterized Manifolds}

\par\noindent Now, consider paramterized manifolds. Let $\alpha:U^{\ptxt{osso}\R^{k}}\to{}Y\eqdef\alpha(U)\subset\R^{n}$, and let $\omega$ be a $k$-form on a neighborhood of $Y$. Then $\int_{Y_{\alpha}}\omega=\int_{U}\alpha^{*}\omega$.\n

\par\noindent What if we reparameterize with $\tilde{\alpha}:V^{\ptxt{osso}\R^{k}}\to{}Y$, with $\Phi$ a diffeomorphic transition map. Then
\[
\int_{Y_{\tilde{\alpha}}}\omega=\int_{V}\tilde{\alpha}^{*}\omega=\pm\int_{U}\Phi^{*}\tilde{\alpha}^{*}\omega=I\int_{Y_{\alpha}}\omega
\]

\par\noindent Where $I$ is positive if $\deg{}D\Phi>0$, and negative if $\det{}D\Phi<0$.\n

\subsection{Integrating $k$-forms over Manifolds}

\par\noindent Let $M$ be a compact $k$-manifold. We want to find $\int_{M}\omega$.\n

\par\noindent Strategy: use partitions of unity to write $\omega=\omega_{1}+\cdots+\omega_{N}$ \st{} $\supp{}\omega_{j}\subseteq{}V_{j}$ with $\alpha_{j}:U_{j}\to{}V_{j}$ coordinate patch.\n

\par\noindent Then set $\displaystyle\int_{M}\omega=\int_{(V_{1})_{\alpha_{1}}}\omega+\cdots+\int_{(V_{N})_{\alpha_{N}}}\omega_{N}$.

\end{document}