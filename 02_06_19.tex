\documentclass[10pt,letterpaper]{article}
\usepackage[utf8]{inputenc}
\usepackage[intlimits]{amsmath}
\usepackage{amsfonts}
\usepackage{amssymb}
\usepackage{ragged2e}
\usepackage[letterpaper, margin=1in]{geometry}
\usepackage{graphicx}
\usepackage{cancel}
\usepackage{mathtools}
\usepackage{tabularx}
\usepackage{arydshln}
\usepackage{tensor}
\usepackage{array}
\usepackage{xcolor}
\usepackage[boxed]{algorithm}
\usepackage[noend]{algpseudocode}
\usepackage{listings}
\usepackage{textcomp}
\usepackage[pdf,tmpdir,singlefile]{graphviz}
\usepackage{mathrsfs}
\usepackage{bbm}
\usepackage{tikz}
\usepackage{enumitem}
\usepackage{arydshln}
\usepackage{relsize}
\usepackage{multicol}

%%%%%%%%%%%%%%%%%%%%%%%%%%%%%
% Formatting commands
%%%%%%%%%%%%%%%%%%%%%%%%%%%%%
\newcommand{\n}{\hfill\break}
\newcommand{\up}{\vspace{-\baselineskip}}
\newcommand{\lemma}[1]{\par\noindent\settowidth{\hangindent}{\textbf{Lemma: }}\textbf{Lemma: }#1}
\newcommand{\defn}[1]{\par\noindent\settowidth{\hangindent}{\textbf{Defn: }}\textbf{Defn: }#1\n}
\newcommand{\thm}[1]{\par\noindent\settowidth{\hangindent}{\textbf{Thm: }}\textbf{Thm: }#1\n}
\newcommand{\prop}[1]{\par\noindent\settowidth{\hangindent}{\textbf{Prop: }}\textbf{Prop: }#1\n}
\newcommand{\cor}[1]{\par\noindent\settowidth{\hangindent}{\textbf{Cor: }}\textbf{Cor: }#1\n}
\newcommand{\ex}[1]{\par\noindent\settowidth{\hangindent}{\textbf{Ex: }}\textbf{Ex: }#1\n}
\newcommand{\exer}[1]{\par\noindent\settowidth{\hangindent}{\textbf{Exer: }}\textbf{Exer: }#1\n}
\newcommand{\proven}{\;$\square$\n}
\newcommand{\problem}[1]{\par\noindent{#1}\n}
\newcommand{\problempart}[2]{\par\noindent\indent{}\settowidth{\hangindent}{\textbf{(#1)} \indent{}}\textbf{(#1)} #2\n}
\newcommand{\ptxt}[1]{\textrm{\textnormal{#1}}}
\newcommand{\inlineeq}[1]{\centerline{$\displaystyle #1$}}
\newcommand{\pageline}{\noindent\rule{\textwidth}{0.1pt}}

%%%%%%%%%%%%%%%%%%%%%%%%%%%%%
% Math commands
%%%%%%%%%%%%%%%%%%%%%%%%%%%%%
% Set Theory
\newcommand{\card}[1]{\left|#1\right|}
\newcommand{\set}[1]{\left\{#1\right\}}
\newcommand{\setmid}{\;\middle|\;}
\newcommand{\ps}[1]{\mathcal{P}\left(#1\right)}
\newcommand{\pfinite}[1]{\mathcal{P}^{\ptxt{finite}}\left(#1\right)}
\newcommand{\naturals}{\mathbb{N}}
\newcommand{\N}{\naturals}
\newcommand{\integers}{\mathbb{Z}}
\newcommand{\Z}{\integers}
\newcommand{\rationals}{\mathbb{Q}}
\newcommand{\Q}{\rationals}
\newcommand{\reals}{\mathbb{R}}
\newcommand{\R}{\reals}
\newcommand{\complex}{\mathbb{C}}
\newcommand{\C}{\complex}
\newcommand{\halfPlane}{\mathbb{H}}
\let\H\relax
\newcommand{\H}{\halfPlane}
\newcommand{\comp}{^{\complement}}
\DeclareMathOperator{\Hom}{Hom}
\newcommand{\Ind}{\mathbbm{1}}
\newcommand{\cut}{\setminus}
\DeclareMathOperator{\elem}{elem}

% Graph Theory
\let\deg\relax
\DeclareMathOperator{\deg}{deg}
\newcommand{\degp}{\ptxt{deg}^{+}}
\newcommand{\degn}{\ptxt{deg}^{-}}
\newcommand{\precdot}{\mathrel{\ooalign{$\prec$\cr\hidewidth\hbox{$\cdot\mkern0.5mu$}\cr}}}
\newcommand{\succdot}{\mathrel{\ooalign{$\cdot\mkern0.5mu$\cr\hidewidth\hbox{$\succ$}\cr\phantom{$\succ$}}}}
\DeclareMathOperator{\cl}{cl}
\DeclareMathOperator{\affdim}{affdim}

% Probability
\newcommand{\Prob}{\mathbb{P}}
\newcommand{\Avg}{\mathbb{E}}

% Standard Math
\newcommand{\inv}{^{-1}}
\newcommand{\abs}[1]{\left|#1\right|}
\newcommand{\ceil}[1]{\left\lceil{}#1\right\rceil{}}
\newcommand{\floor}[1]{\left\lfloor{}#1\right\rfloor{}}
\newcommand{\conj}[1]{\overline{#1}}
\newcommand{\of}{\circ}
\newcommand{\tri}{\triangle}
\newcommand{\inj}{\hookrightarrow}
\newcommand{\surj}{\twoheadrightarrow}
\newcommand{\ndiv}{\nmid}
\renewcommand{\epsilon}{\varepsilon}
\newcommand{\divides}{\mid}
\newcommand{\ndivides}{\nmid}
\DeclareMathOperator{\lcm}{lcm}
\DeclareMathOperator{\sgn}{sgn}
\newcommand{\map}[4]{\!\!\!\begin{array}[t]{rcl}#1 & \!\!\!\!\to & \!\!\!\!#2\\ #3 & \!\!\!\!\mapsto & \!\!\!\!#4\end{array}}
\newcommand{\bigsum}[2]{\smashoperator[lr]{\sum_{\scalebox{#1}{$#2$}}}}

% Linear Algebra
\newcommand{\Id}{\textrm{\textnormal{Id}}}
\newcommand{\im}{\textrm{\textnormal{im}}}
\newcommand{\norm}[1]{\abs{\abs{#1}}}
\newcommand{\tpose}{^{T}}
\newcommand{\iprod}[1]{\left<#1\right>}
\DeclareMathOperator{\trace}{tr}
\newcommand{\chgBasMat}[3]{\!\!\tensor*[_{#1}]{\left[#2\right]}{_{#3}}}
\newcommand{\vecBas}[2]{\tensor*[]{\left[#1\right]}{_{#2}}}
\DeclareMathOperator{\GL}{GL}
\DeclareMathOperator{\Mat}{Mat}
\DeclareMathOperator{\vspan}{span}
\DeclareMathOperator{\rank}{rank}
\newcommand{\V}[1]{\vec{#1}}
\DeclareMathOperator{\proj}{proj}
\DeclareMathOperator{\compProj}{comp}

% Multilinear Algebra
\newcommand{\Lsym}{\L}
\let\L\relax
\DeclareMathOperator{\L}{\mathscr{L}}
\DeclareMathOperator{\A}{\mathcal{A}}
\DeclareMathOperator{\Alt}{Alt}
\DeclareMathOperator{\Sym}{Sym}
\newcommand{\ot}{\otimes}
\newcommand{\ox}{\otimes}
\DeclareMathOperator{\asc}{asc}
\DeclareMathOperator{\asSet}{set}
\DeclareMathOperator{\sort}{sort}
\DeclareMathOperator{\ringA}{\mathring{A}}

% Topology
\newcommand{\closure}[1]{\overline{#1}}
\newcommand{\uball}{\mathcal{U}}
\DeclareMathOperator{\Int}{Int}
\DeclareMathOperator{\Ext}{Ext}
\DeclareMathOperator{\Bd}{Bd}
\DeclareMathOperator{\rInt}{rInt}
\DeclareMathOperator{\ch}{ch}
\DeclareMathOperator{\ah}{ah}
\newcommand{\Tau}{\mathlarger{\mathlarger{\mathlarger{\mathlarger{\tau}}}}}

% Analysis
\DeclareMathOperator{\Graph}{Graph}
\DeclareMathOperator{\epi}{epi}
\DeclareMathOperator{\hypo}{hypo}
\DeclareMathOperator{\supp}{supp}
\newcommand{\lint}[2]{\underset{#1}{\overset{#2}{{\color{black}\underline{{\color{white}\overline{{\color{black}\int}}\color{black}}}}}}}
\newcommand{\uint}[2]{\underset{#1}{\overset{#2}{{\color{white}\underline{{\color{black}\overline{{\color{black}\int}}\color{black}}}}}}}
\newcommand{\alignint}[2]{\underset{#1}{\overset{#2}{{\color{white}\underline{{\color{white}\overline{{\color{black}\int}}\color{black}}}}}}}
\newcommand{\extint}{\ptxt{ext}\int}
\newcommand{\extalignint}[2]{\ptxt{ext}\alignint{#1}{#2}}
\newcommand{\conv}{\ast}

% Proofs
\newcommand{\st}{s.t.}
\newcommand{\unique}{!}
\newcommand{\iffdef}{\overset{\ptxt{def}}{\Leftrightarrow}}
\newcommand{\eqdef}{\overset{\ptxt{def}}{=}}
\newcommand{\eqVertical}{\rotatebox[origin=c]{90}{=}}
\newcommand{\mapsfrom}{\mathrel{\reflectbox{\ensuremath{\mapsto}}}}
\newcommand{\mapsdown}{\rotatebox[origin=c]{-90}{$\mapsto$}\mkern2mu}
\newcommand{\mapsup}{\rotatebox[origin=c]{90}{$\mapsto$}\mkern2mu}

% Brackets
\newcommand{\paren}[1]{\left(#1\right)}
\renewcommand{\brack}[1]{\left[#1\right]}
\renewcommand{\brace}[1]{\left\{#1\right\}}
\newcommand{\ang}[1]{\left<#1\right>}

% Algorithms
\algrenewcommand{\algorithmiccomment}[1]{\hskip 1em \texttt{// #1}}
\algrenewcommand\algorithmicrequire{\textbf{Input:}}
\algrenewcommand\algorithmicensure{\textbf{Output:}}
\newcommand{\parSymbol}{\P}
\renewcommand{\P}{\ptxt{\textbf{P}}}
\newcommand{\NP}{\ptxt{\textbf{NP}}}
\newcommand{\NPC}{\ptxt{\textbf{NP-Complete}}}
\newcommand{\NPH}{\ptxt{\textbf{NP-Hard}}}
\newcommand{\EXP}{\ptxt{\textbf{EXP}}}

%%%%%%%%%%%%%%%%%%%%%%%%%%%%%
% Other commands
%%%%%%%%%%%%%%%%%%%%%%%%%%%%%
\newcommand{\flag}[1]{\textbf{\textcolor{red}{#1}}}
\newcommand{\uSym}{\u}
\let\u\relax
\newcommand{\u}[1]{\underline{#1}}
\newcommand{\bSym}{\b}
\let\b\relax
\newcommand{\b}[1]{\textbf{#1}}
\newcommand{\iSym}{\i}
\let\i\relax
\newcommand{\i}[1]{\textit{#1}}

%%%%%%%%%%%%%%%%%%%%%%%%%%%%%
% Make l's curvy in math environments
%%%%%%%%%%%%%%%%%%%%%%%%%%%%%
\mathcode`l="8000
\begingroup
\makeatletter
\lccode`\~=`\l
\DeclareMathSymbol{\lsb@l}{\mathalpha}{letters}{`l}
\lowercase{\gdef~{\ifnum\the\mathgroup=\m@ne \ell \else \lsb@l \fi}}%
\endgroup

\newcommand{\B}{
    \begin{tikzpicture}
    \filldraw [fill=red, draw=black] (0, 0) rectangle (0.37, 0.45);
    \draw [line width=0.5mm, white ] (0.1,0.08) -- (0.1,0.38);
    \draw[line width=0.5mm, white ] (0.1, 0.35) .. controls (0.2, 0.35) and (0.4, 0.2625) .. (0.1, 0.225);
    \draw[line width=0.5mm, white ] (0.1, 0.225) .. controls (0.2, 0.225) and (0.4, 0.1625) .. (0.1, 0.1);
    \end{tikzpicture}
}

\author{Thomas Cohn}
\title{Manifold Orientation}
\date{2/6/19} % Can also use \today

\begin{document}
\maketitle
\setlength\RaggedRightParindent{\parindent}
\RaggedRight

\defn{An \u{orientation} of manifold $M$ is a division of connected coordinate patches into Group A, Group B \st{} $\det{}D(\alpha_{2}\inv\of\alpha_{1})$ is positive for $\alpha_{1},\alpha_{2}$ in the same group and negative for $\alpha_{1},\alpha_{2}$ in different groups.}

\par\noindent By convention, Group A patches are ``orientation-preserving'', and Group B patches are ``orientation-reversing''.\n

\par\noindent But how do we specify orientation? It can be awkward. And we want a ``default'' orientation in certain situations.\n

\prop{Given $M$ with an orientation, then $\exists\omega$ $k$-form on a neighborhood of $M$ \st{} $\alpha^{*}\omega$ is a positive multiple of $dx_{1}\wedge\cdots{}dx_{k}$ when $\alpha$ is an orientation-preserving coordinate patch, and negative when $\alpha$ is an orientation-reversing coordinate patch.\n
Proof: HW5\n}

\par\noindent Given $\omega$ $k$-form on a neighborhood of a compact oriented manifold $M$. To define $\int_{M}\omega$, use partition of unity to write $\omega=\omega_{1}+\cdots+\omega_{N}$ \st{} $\supp\omega_{j}\subseteq{}V_{j}$ for some orientation-preserving coordinate patch $\alpha_{j}:U_{j}\to{}V_{j}$.\n

\par\noindent Claim: This does not depend on choices made.\n
Proof: same as for $\int_{M}f\,dV$ on 1/11.\n

\par\noindent Immediate result: $\displaystyle\int_{M}c\omega=c\int_{M}\omega$ and $\displaystyle\int_{M}\omega_{1}+\omega_{2}=\int_{M}\omega_{1}+\int_{M}\omega_{2}$.\n

\par\noindent Reverse orientation: replace $\displaystyle\int_{M}\omega$ by $\displaystyle-\int_{M}\omega$.\n

\defn{The inverse of a coordinate patch is called a \u{coordinate chart}.}

\par\noindent Suppose we're just given $U\overset{\alpha}{\to}V\overset{\beta}{\to}\R^{k}$.\n

\par\noindent Check: points at which $\beta$ fails to be a coordinate patch are points at which $\alpha^{*}(d\beta_{1}\wedge\cdots\wedge{}d\beta_{k})=0$. Such points contribute nothing to the integral.\n

\newpage

\ex{Manifold $M$ in the shape of a chef's hat (open at the bottom). Then\n
$\displaystyle\int_{M}dx\wedge{}dy=\pi$\n
$\displaystyle\int_{M}dy\wedge{}dz=0$\n
$\displaystyle\int_{M}dx\wedge{}dz=0$.\n}

\ex{Manifold $M$ is the unit sphere in $\R^{3}$. Then\n
$\displaystyle\int_{M}dx\wedge{}dy=0$\n
$\displaystyle\int_{M}dx\wedge{}dz=0$\n
$\displaystyle\int_{M}dy\wedge{}dz=0$\n
\n
$\displaystyle\int_{M}x\,dx\wedge{}dy=0$\n
$\displaystyle\int_{M}y\,dx\wedge{}dy=0$\n
$\displaystyle\int_{M}z\,dx\wedge{}dy=z\smashoperator{\int_{\set{x^{2}+y^{2}<1}}}\sqrt{1-x^{2}-y^{2}}=4\pi\int_{0}^{1}\sqrt{1-r^{2}}r\,dr=\cdots=\frac{4\pi}{3}\ne{}0$.\n}

\par\noindent Goal: $\displaystyle\int_{M}d\omega=\int_{\partial{}M}\omega$. To get this, we need an orientation on $M$ to imply an orientation on $\partial{}M$.\n

\par\noindent For $M$ manifold-with-boundary, $U_{1},U_{2}\subseteq\H^{k}$, $\alpha_{i}:U_{i}\to{}M$ orientation-preserving coordinate patches, $V_{i}\eqdef\alpha_{i}[U_{i}]$, and $\varphi=\alpha_{2}\inv\of\alpha_{1}$ orientation-preserving transition map with $\varphi_{k}\ge{}0$. From 1/9, we know $\varphi_{k}=0$ when $x_{k}=0$.\n

\par\noindent Consider patches for $\partial{}M$, $\displaystyle\tilde{\alpha}_{1}\paren{\begin{array}{c}x_{1}\\ \vdots\\ x_{k-1}\end{array}}=\alpha_{j}\paren{\begin{array}{c}x_{1}\\ \vdots\\ x_{k-1}\\ 0\end{array}}$, $\tilde{\varphi}=\tilde{\alpha_{2}}\inv\of\tilde{\alpha_{1}}$.\n

\par\noindent When $x_{k}=0$,
\[
D\varphi=\paren{\begin{array}{c:c}
 & \\
D\tilde{\varphi} & \\
 & \\
\underbrace{\frac{\partial\varphi_{k}}{\partial{}x_{1}}\;\;\cdots\;\;\frac{\partial\varphi_{k}}{\partial{}x_{k-1}}}_{\ptxt{all }0} & \frac{\partial\varphi_{k}}{\partial{}x_{k}}\\
\end{array}}
\]

\par\noindent $\displaystyle\frac{\partial\varphi_{k}}{\partial{}x_{k}}=\lim_{\substack{h\to{}0\\ (h>0)}}\frac{\overbrace{\varphi_{k}(\vec{p}+h\vec{e_{k}})}^{\ge{}0}-\cancelto{0}{\varphi_{k}(\vec{p})}}{h}$. This is non-negative fraction, so it's a non-negative limit, so
\[
\underbrace{\deg{}D\varphi}_{\ptxt{positive}}=\det{}D\tilde{\varphi}\cdot\frac{\partial\varphi_{k}}{\partial{}x_{k}}
\]

\par\noindent Because $\det{}D\varphi$ is positive, and $\frac{\partial\varphi_{k}}{\partial{}x_{k}}>0$, we must have $\det{}D\tilde{\varphi}$ also positive.\n
So we get an orientation on $\partial{}M$ by declaring $\tilde{\alpha_{j}}$ to be orientation-preserving.

\end{document}