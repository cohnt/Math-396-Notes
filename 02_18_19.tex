\documentclass[10pt,letterpaper]{article}
\usepackage[utf8]{inputenc}
\usepackage[intlimits]{amsmath}
\usepackage{amsfonts}
\usepackage{amssymb}
\usepackage{ragged2e}
\usepackage[letterpaper, margin=1in]{geometry}
\usepackage{graphicx}
\usepackage{cancel}
\usepackage{mathtools}
\usepackage{tabularx}
\usepackage{arydshln}
\usepackage{tensor}
\usepackage{array}
\usepackage{xcolor}
\usepackage[boxed]{algorithm}
\usepackage[noend]{algpseudocode}
\usepackage{listings}
\usepackage{textcomp}
\usepackage[pdf,tmpdir,singlefile]{graphviz}
\usepackage{mathrsfs}
\usepackage{bbm}
\usepackage{tikz}
\usepackage{enumitem}
\usepackage{arydshln}
\usepackage{relsize}
\usepackage{multicol}

%%%%%%%%%%%%%%%%%%%%%%%%%%%%%
% Formatting commands
%%%%%%%%%%%%%%%%%%%%%%%%%%%%%
\newcommand{\n}{\hfill\break}
\newcommand{\up}{\vspace{-\baselineskip}}
\newcommand{\lemma}[1]{\par\noindent\settowidth{\hangindent}{\textbf{Lemma: }}\textbf{Lemma: }#1}
\newcommand{\defn}[1]{\par\noindent\settowidth{\hangindent}{\textbf{Defn: }}\textbf{Defn: }#1\n}
\newcommand{\thm}[1]{\par\noindent\settowidth{\hangindent}{\textbf{Thm: }}\textbf{Thm: }#1\n}
\newcommand{\prop}[1]{\par\noindent\settowidth{\hangindent}{\textbf{Prop: }}\textbf{Prop: }#1\n}
\newcommand{\cor}[1]{\par\noindent\settowidth{\hangindent}{\textbf{Cor: }}\textbf{Cor: }#1\n}
\newcommand{\ex}[1]{\par\noindent\settowidth{\hangindent}{\textbf{Ex: }}\textbf{Ex: }#1\n}
\newcommand{\exer}[1]{\par\noindent\settowidth{\hangindent}{\textbf{Exer: }}\textbf{Exer: }#1\n}
\newcommand{\proven}{\;$\square$\n}
\newcommand{\problem}[1]{\par\noindent{#1}\n}
\newcommand{\problempart}[2]{\par\noindent\indent{}\settowidth{\hangindent}{\textbf{(#1)} \indent{}}\textbf{(#1)} #2\n}
\newcommand{\ptxt}[1]{\textrm{\textnormal{#1}}}
\newcommand{\inlineeq}[1]{\centerline{$\displaystyle #1$}}
\newcommand{\pageline}{\noindent\rule{\textwidth}{0.1pt}}

%%%%%%%%%%%%%%%%%%%%%%%%%%%%%
% Math commands
%%%%%%%%%%%%%%%%%%%%%%%%%%%%%
% Set Theory
\newcommand{\card}[1]{\left|#1\right|}
\newcommand{\set}[1]{\left\{#1\right\}}
\newcommand{\setmid}{\;\middle|\;}
\newcommand{\ps}[1]{\mathcal{P}\left(#1\right)}
\newcommand{\pfinite}[1]{\mathcal{P}^{\ptxt{finite}}\left(#1\right)}
\newcommand{\naturals}{\mathbb{N}}
\newcommand{\N}{\naturals}
\newcommand{\integers}{\mathbb{Z}}
\newcommand{\Z}{\integers}
\newcommand{\rationals}{\mathbb{Q}}
\newcommand{\Q}{\rationals}
\newcommand{\reals}{\mathbb{R}}
\newcommand{\R}{\reals}
\newcommand{\complex}{\mathbb{C}}
\newcommand{\C}{\complex}
\newcommand{\halfPlane}{\mathbb{H}}
\let\H\relax
\newcommand{\H}{\halfPlane}
\newcommand{\comp}{^{\complement}}
\DeclareMathOperator{\Hom}{Hom}
\newcommand{\Ind}{\mathbbm{1}}
\newcommand{\cut}{\setminus}
\DeclareMathOperator{\elem}{elem}

% Graph Theory
\let\deg\relax
\DeclareMathOperator{\deg}{deg}
\newcommand{\degp}{\ptxt{deg}^{+}}
\newcommand{\degn}{\ptxt{deg}^{-}}
\newcommand{\precdot}{\mathrel{\ooalign{$\prec$\cr\hidewidth\hbox{$\cdot\mkern0.5mu$}\cr}}}
\newcommand{\succdot}{\mathrel{\ooalign{$\cdot\mkern0.5mu$\cr\hidewidth\hbox{$\succ$}\cr\phantom{$\succ$}}}}
\DeclareMathOperator{\cl}{cl}
\DeclareMathOperator{\affdim}{affdim}

% Probability
\newcommand{\Prob}{\mathbb{P}}
\newcommand{\Avg}{\mathbb{E}}
\DeclareMathOperator{\Var}{Var}
\DeclareMathOperator{\cov}{cov}

% Standard Math
\newcommand{\inv}{^{-1}}
\newcommand{\abs}[1]{\left|#1\right|}
\newcommand{\ceil}[1]{\left\lceil{}#1\right\rceil{}}
\newcommand{\floor}[1]{\left\lfloor{}#1\right\rfloor{}}
\newcommand{\conj}[1]{\overline{#1}}
\newcommand{\of}{\circ}
\newcommand{\tri}{\triangle}
\newcommand{\inj}{\hookrightarrow}
\newcommand{\surj}{\twoheadrightarrow}
\newcommand{\ndiv}{\nmid}
\renewcommand{\epsilon}{\varepsilon}
\newcommand{\divides}{\mid}
\newcommand{\ndivides}{\nmid}
\DeclareMathOperator{\lcm}{lcm}
\DeclareMathOperator{\sgn}{sgn}
\newcommand{\map}[4]{\!\!\!\begin{array}[t]{rcl}#1 & \!\!\!\!\to & \!\!\!\!#2\\ #3 & \!\!\!\!\mapsto & \!\!\!\!#4\end{array}}
\newcommand{\bigsum}[2]{\smashoperator[lr]{\sum_{\scalebox{#1}{$#2$}}}}

% Linear Algebra
\newcommand{\Id}{\textrm{\textnormal{Id}}}
\newcommand{\im}{\textrm{\textnormal{im}}}
\newcommand{\norm}[1]{\abs{\abs{#1}}}
\newcommand{\tpose}{^{T}}
\newcommand{\iprod}[1]{\left<#1\right>}
\DeclareMathOperator{\trace}{tr}
\newcommand{\chgBasMat}[3]{\!\!\tensor*[_{#1}]{\left[#2\right]}{_{#3}}}
\newcommand{\vecBas}[2]{\tensor*[]{\left[#1\right]}{_{#2}}}
\DeclareMathOperator{\GL}{GL}
\DeclareMathOperator{\Mat}{Mat}
\DeclareMathOperator{\vspan}{span}
\DeclareMathOperator{\rank}{rank}
\newcommand{\V}[1]{\vec{#1}}
\DeclareMathOperator{\proj}{proj}
\DeclareMathOperator{\compProj}{comp}
\DeclareMathOperator{\row}{row}

% Multilinear Algebra
\newcommand{\Lsym}{\L}
\let\L\relax
\DeclareMathOperator{\L}{\mathscr{L}}
\DeclareMathOperator{\A}{\mathcal{A}}
\DeclareMathOperator{\Alt}{Alt}
\DeclareMathOperator{\Sym}{Sym}
\newcommand{\ot}{\otimes}
\newcommand{\ox}{\otimes}
\DeclareMathOperator{\asc}{asc}
\DeclareMathOperator{\asSet}{set}
\DeclareMathOperator{\sort}{sort}
\DeclareMathOperator{\ringA}{\mathring{A}}

% Topology
\newcommand{\closure}[1]{\overline{#1}}
\newcommand{\uball}{\mathcal{U}}
\DeclareMathOperator{\Int}{Int}
\DeclareMathOperator{\Ext}{Ext}
\DeclareMathOperator{\Bd}{Bd}
\DeclareMathOperator{\rInt}{rInt}
\DeclareMathOperator{\ch}{ch}
\DeclareMathOperator{\ah}{ah}
\newcommand{\Tau}{\mathlarger{\mathlarger{\mathlarger{\mathlarger{\tau}}}}}

% Analysis
\DeclareMathOperator{\Graph}{Graph}
\DeclareMathOperator{\epi}{epi}
\DeclareMathOperator{\hypo}{hypo}
\DeclareMathOperator{\supp}{supp}
\newcommand{\lint}[2]{\underset{#1}{\overset{#2}{{\color{black}\underline{{\color{white}\overline{{\color{black}\int}}\color{black}}}}}}}
\newcommand{\uint}[2]{\underset{#1}{\overset{#2}{{\color{white}\underline{{\color{black}\overline{{\color{black}\int}}\color{black}}}}}}}
\newcommand{\alignint}[2]{\underset{#1}{\overset{#2}{{\color{white}\underline{{\color{white}\overline{{\color{black}\int}}\color{black}}}}}}}
\newcommand{\extint}{\ptxt{ext}\int}
\newcommand{\extalignint}[2]{\ptxt{ext}\alignint{#1}{#2}}
\newcommand{\conv}{\ast}
\newcommand{\pd}[2]{\frac{\partial{}#1}{\partial{}#2}}
\newcommand{\del}{\nabla}
\DeclareMathOperator{\grad}{grad}
\DeclareMathOperator{\curl}{curl}
\let\div\relax
\DeclareMathOperator{\div}{div}

% Complex Analysis
\let\Re\relax
\DeclareMathOperator{\Re}{Re}
\let\Im\relax
\DeclareMathOperator{\Im}{Im}

% Proofs
\newcommand{\st}{s.t.}
\newcommand{\unique}{!}
\newcommand{\iffdef}{\overset{\ptxt{def}}{\Leftrightarrow}}
\newcommand{\eqdef}{\overset{\ptxt{def}}{=}}
\newcommand{\eqVertical}{\rotatebox[origin=c]{90}{=}}
\newcommand{\mapsfrom}{\mathrel{\reflectbox{\ensuremath{\mapsto}}}}
\newcommand{\mapsdown}{\rotatebox[origin=c]{-90}{$\mapsto$}\mkern2mu}
\newcommand{\mapsup}{\rotatebox[origin=c]{90}{$\mapsto$}\mkern2mu}
\newcommand{\from}{\!\mathrel{\reflectbox{\ensuremath{\to}}}}

% Brackets
\newcommand{\paren}[1]{\left(#1\right)}
\renewcommand{\brack}[1]{\left[#1\right]}
\renewcommand{\brace}[1]{\left\{#1\right\}}
\newcommand{\ang}[1]{\left<#1\right>}

% Algorithms
\algrenewcommand{\algorithmiccomment}[1]{\hskip 1em \texttt{// #1}}
\algrenewcommand\algorithmicrequire{\textbf{Input:}}
\algrenewcommand\algorithmicensure{\textbf{Output:}}
\newcommand{\parSymbol}{\P}
\renewcommand{\P}{\ptxt{\textbf{P}}}
\newcommand{\NP}{\ptxt{\textbf{NP}}}
\newcommand{\NPC}{\ptxt{\textbf{NP-Complete}}}
\newcommand{\NPH}{\ptxt{\textbf{NP-Hard}}}
\newcommand{\EXP}{\ptxt{\textbf{EXP}}}

%%%%%%%%%%%%%%%%%%%%%%%%%%%%%
% Other commands
%%%%%%%%%%%%%%%%%%%%%%%%%%%%%
\newcommand{\flag}[1]{\textbf{\textcolor{red}{#1}}}
\newcommand{\uSym}{\u}
\let\u\relax
\newcommand{\u}[1]{\underline{#1}}
\newcommand{\bSym}{\b}
\let\b\relax
\newcommand{\b}[1]{\textbf{#1}}
\newcommand{\iSym}{\i}
\let\i\relax
\newcommand{\i}[1]{\textit{#1}}

%%%%%%%%%%%%%%%%%%%%%%%%%%%%%
% Make l's curvy in math environments
%%%%%%%%%%%%%%%%%%%%%%%%%%%%%
\mathcode`l="8000
\begingroup
\makeatletter
\lccode`\~=`\l
\DeclareMathSymbol{\lsb@l}{\mathalpha}{letters}{`l}
\lowercase{\gdef~{\ifnum\the\mathgroup=\m@ne \ell \else \lsb@l \fi}}%
\endgroup

\newcommand{\B}{
    \begin{tikzpicture}
    \filldraw [fill=red, draw=black] (0, 0) rectangle (0.37, 0.45);
    \draw [line width=0.5mm, white ] (0.1,0.08) -- (0.1,0.38);
    \draw[line width=0.5mm, white ] (0.1, 0.35) .. controls (0.2, 0.35) and (0.4, 0.2625) .. (0.1, 0.225);
    \draw[line width=0.5mm, white ] (0.1, 0.225) .. controls (0.2, 0.225) and (0.4, 0.1625) .. (0.1, 0.1);
    \end{tikzpicture}
}

\author{Professor David Barrett\\ \small\textit{Transcribed by Thomas Cohn}}
\title{Complex Numbers}
\date{2/18/19} % Can also use \today

\begin{document}
\maketitle
\setlength\RaggedRightParindent{\parindent}
\RaggedRight

\begin{center}
\begin{tabular}{l|l}
	Exterior Calculus in $\R^{3}$ & Vector Calculus in $\R^{3}$\\ \hline
	$\left.\begin{array}{r}0\ptxt{-form}\\ 3\ptxt{-form}\end{array}\right\}$ & Scalar Function\\
	$\left.\begin{array}{r}1\ptxt{-form}\\ 2\ptxt{-form}\end{array}\right\}$ & Vector Field\\
	$d$ & $\del$\\
	$d(0\ptxt{-form})$ & $\grad{}f=\del{}f$\\
	$d(1\ptxt{-form})$ & $\curl\vec{F}=\del\times\vec{F}$\\
	$d(2\ptxt{-form})$ & $\div\vec{F}=\iprod{\del,\vec{F}}$
\end{tabular}
\end{center}

\par\noindent $\iprod{\curl\vec{F},\vec{N}}$ is the rotation of $\vec{F}$ in the plane perpendicular to $\vec{N}$ based on $\int_{M}d\omega$.\n

\par\noindent For $M$ oriented $2$-manifold in $\R^{3}$, $\omega$ $2$-form
\[
\int_{M}\omega=\int_{\vec{p}\in{}M}\omega(\vec{p})(\vec{v_{1}},\vec{v_{2}})\,ds=\int\vec{F}\cdot\vec{N}\,ds
\]
where $\vec{F}$ is the vector field corresponding to $\omega$ and $(\vec{v_{1}},\vec{v_{2}},\vec{N})$ form a positively-oriented orthonormal basis.\n

\par\noindent We can thus interpret $\vec{F}$ as the velocity of a fluid with unit density, so $\int_{M}\omega$ is the flux of $\vec{F}$ across $M$. $\vec{F}\cdot\vec{N}>0$ implies a positive flow, $\vec{F}\cdot\vec{N}<0$ implies a negative flow.\n

\par\noindent If the fluid has density $\rho$, then the fluid crosses $M$ at rate (flux of $\rho\vec{F}$).\n

\ex{For compact $3$-manifold $U\subset\R^{3}$, the flow out of $U$ is equal to the flux of $\rho\vec{F}$ across $\partial{}U$, which according to Stokes', is $\int_{U}\div\vec{F}$.\n
Additionally, we know the flow out of $U$ is $-\pd{}{t}\int_{U}\rho=-\int\pd{\rho}{t}$ by Leibniz. Since this is true for all $U$, we have $\pd{\rho}{t}=-\div(\rho\vec{F})$.}

\par\noindent Thus, if $\div(\rho\vec{F})>0$, the fluid is expanding, and if $\div(\rho\vec{F})<0$, the fuild is contracting. If $\rho$ is constant in space and time, then the fluid is incompressible, $\div(\rho\vec{F})=0$.\n

\par\noindent Now, we will talk about Euler's equations for non-viscous (i.e. no fluid friction), incompressible fluids.\n
$\div\vec{F}=0$\n
$\pd{\curl\vec{F}}{t}+\curl((\curl\vec{F})\times\vec{F})=\vec{0}$.\n

\par\noindent We can upgrade to include viscosity, and we get Navier-Stokes'\n

\par\noindent Special Case of Euler's equations: $\div\vec{F}=\vec{0}$, $\curl\vec{F}=\vec{0}$ (incompressible and irrotational).\n
Even more special case: $\vec{F}=\paren{\begin{array}{c}\alpha(x,y)\\ \beta(x,y)\\ 0\end{array}}$. Get $\pd{\alpha}{x}+\pd{\beta}{y}=0$ and $\pd{\beta}{x}-\pd{\alpha}{y}=0$, i.e., $\pd{\alpha}{x}=-\pd{\beta}{y}$, $\pd{\alpha}{y}=\pd{\beta}{x}$.\n

\section*{Complex Integrands}

\defn{Consider $f:A\to{}C$, where $f=u+iv$ for $\R$-valued $u$ and $v$. We say $u\eqdef\Re{}f$ and $v\eqdef\Im{}f$.\n
\inlineeq{
	\int_{A}f\eqdef\int_{A}u+i\int_{A}v
}}

\par\noindent Use the extended integral when $\extint{}u_{+}$, $\extint{}u_{-}$, $\extint{}v_{+}$, and $\extint{}v_{-}$ are all finite.\n

\par\noindent We also have $\displaystyle\int_{M}f\,dV=\int_{M}u\,dV+i\int_{M}v\,dV$.\n

\exer{$\displaystyle\lambda\in\C\Rightarrow\int_{A}\lambda{}f=\lambda\int_{A}f$}

\prop{$\displaystyle\abs{\int_{A}f}\le\int_{A}\abs{f}$.\n
Proof: pick $\theta$ such that $\norm{\int{}f}=e^{i\theta}\int{}f$.\n
Then $\abs{\int{}f}=\Re\int{}e^{i\theta}f=\int\Re(e^{i\theta}f)\le\int\abs{\Re(e^{i\theta}f)}\le\int\abs{e^{i\theta}f}=\int\abs{f}$.\proven}

\section*{$\C$-valued $k$-forms}

\par\noindent Recall that a $k$-form on $A\subseteq\R^{n}$ is $\displaystyle\omega=\sum_{\mathclap{I\in\set{1,\ldots,n}^{k}\asc}}b_{I}(\vec{x})\Psi_{I}$ for $b_{I}\in{}C(A,\R)$.\n

\par\noindent A $\C$-valued $k$-form on $A$ is defined exactly the same, except we have each $b_{I}\in{}C(A,\C)$.\n

\par\noindent An $\R$-valued $k$-form maps $\displaystyle{}A\to\A^{k}(\R^{n})=\set{f:\smash{\underbrace{\R^{n}\times\cdots\times\R^{n}}_{k}}\to\R:f\ptxt{ is $\R$-multilinear and alternating}}$.\n

\par\noindent Likewise, a $\C$-valued $k$-form maps $\displaystyle{}A\to\A^{k}_{\C}(\R^{n})=\set{f:\smash{\underbrace{\R^{n}\times\cdots\times\R^{n}}_{k}}\to\C:f\ptxt{ is $\R$-multilinear and alternating}}$.\n

\defn{If $k=n$, $\omega=b_{(1,\ldots,k)}\Psi_{(1,\ldots,k)}$, then $\displaystyle\int_{A}\omega\eqdef\int_{A}b_{(1,\ldots,k)}$}

\par\noindent Rules for $\wedge$, $\alpha^{*}$, and $d$ all carry over.\n
For $\alpha:A^{\ptxt{osso}\R^{k}}\to{}Y\subseteq\R^{n}$, $Y_{\alpha}$ parameterized $k$-manifold, $\omega$ $\C$-valued $k$-form, $\int_{Y_{\alpha}}\omega=\int_{A}\alpha^{*}\omega=\cdots$.\n
We also still have $\int_{Y_{\alpha}}\lambda\omega=\lambda\int_{Y_{\alpha}}\omega$.\n

\par\noindent Consider $k=1$, $\omega:A\to\C^{n}_{\ptxt{row}}$.

\exer{We still have the ML estimate: $\abs{\int_{Y_{\alpha}}\omega}\le\operatorname{length}(Y_{\alpha})\cdot\underset{\vec{x}\in{}Y}{\sup}\norm{\omega(\vec{x})}$}

\defn{Consider $f=u+iv:A^{\ptxt{osso}\R^{k}}\to\C\simeq\R^{2}$. Then $f\in{}C^{r}(A,\C)$ if and only if $u,v\in{}C^{r}(A,\R)$.\n
Also, $D_{j}f=D_{j}u+iD_{j}v$, and $df\eqdef{}du+idv$.}

\par\noindent Now, consider $n=2$, $z=x+iy$. So $\conj{z}=x-iy$, $dz=dx+idy$, and $d\conj{z}=dx-idy$. Thus, we can write
\[
dx=\frac{dz+d\conj{z}}{2}\quad\ptxt{and}\quad{}dy=\frac{dz-d\conj{z}}{2i}
\]

\par\noindent Thus, for $\alpha,\beta,f,g$ $\C$-valued functions, we can correspond $\alpha{}dx+\beta{}dy$ with $fdz+gd\conj{z}$.\n

\par\noindent This leads to another question -- given $f=u+iv$, when is $fdz$ closed?\n
Well, $fdz=(u+iv)(dx+idy)=(udx-vdy)+i(vdx+udy)$.\n
So both $(udx-vdy)$ and $(vdx+udy)$ must be closed separately.

\end{document}