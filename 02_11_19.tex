\documentclass[10pt,letterpaper]{article}
\usepackage[utf8]{inputenc}
\usepackage[intlimits]{amsmath}
\usepackage{amsfonts}
\usepackage{amssymb}
\usepackage{ragged2e}
\usepackage[letterpaper, margin=1in]{geometry}
\usepackage{graphicx}
\usepackage{cancel}
\usepackage{mathtools}
\usepackage{tabularx}
\usepackage{arydshln}
\usepackage{tensor}
\usepackage{array}
\usepackage{xcolor}
\usepackage[boxed]{algorithm}
\usepackage[noend]{algpseudocode}
\usepackage{listings}
\usepackage{textcomp}
\usepackage[pdf,tmpdir,singlefile]{graphviz}
\usepackage{mathrsfs}
\usepackage{bbm}
\usepackage{tikz}
\usepackage{enumitem}
\usepackage{arydshln}
\usepackage{relsize}
\usepackage{multicol}

%%%%%%%%%%%%%%%%%%%%%%%%%%%%%
% Formatting commands
%%%%%%%%%%%%%%%%%%%%%%%%%%%%%
\newcommand{\n}{\hfill\break}
\newcommand{\up}{\vspace{-\baselineskip}}
\newcommand{\lemma}[1]{\par\noindent\settowidth{\hangindent}{\textbf{Lemma: }}\textbf{Lemma: }#1}
\newcommand{\defn}[1]{\par\noindent\settowidth{\hangindent}{\textbf{Defn: }}\textbf{Defn: }#1\n}
\newcommand{\thm}[1]{\par\noindent\settowidth{\hangindent}{\textbf{Thm: }}\textbf{Thm: }#1\n}
\newcommand{\prop}[1]{\par\noindent\settowidth{\hangindent}{\textbf{Prop: }}\textbf{Prop: }#1\n}
\newcommand{\cor}[1]{\par\noindent\settowidth{\hangindent}{\textbf{Cor: }}\textbf{Cor: }#1\n}
\newcommand{\ex}[1]{\par\noindent\settowidth{\hangindent}{\textbf{Ex: }}\textbf{Ex: }#1\n}
\newcommand{\exer}[1]{\par\noindent\settowidth{\hangindent}{\textbf{Exer: }}\textbf{Exer: }#1\n}
\newcommand{\proven}{\;$\square$\n}
\newcommand{\problem}[1]{\par\noindent{#1}\n}
\newcommand{\problempart}[2]{\par\noindent\indent{}\settowidth{\hangindent}{\textbf{(#1)} \indent{}}\textbf{(#1)} #2\n}
\newcommand{\ptxt}[1]{\textrm{\textnormal{#1}}}
\newcommand{\inlineeq}[1]{\centerline{$\displaystyle #1$}}
\newcommand{\pageline}{\noindent\rule{\textwidth}{0.1pt}}

%%%%%%%%%%%%%%%%%%%%%%%%%%%%%
% Math commands
%%%%%%%%%%%%%%%%%%%%%%%%%%%%%
% Set Theory
\newcommand{\card}[1]{\left|#1\right|}
\newcommand{\set}[1]{\left\{#1\right\}}
\newcommand{\setmid}{\;\middle|\;}
\newcommand{\ps}[1]{\mathcal{P}\left(#1\right)}
\newcommand{\pfinite}[1]{\mathcal{P}^{\ptxt{finite}}\left(#1\right)}
\newcommand{\naturals}{\mathbb{N}}
\newcommand{\N}{\naturals}
\newcommand{\integers}{\mathbb{Z}}
\newcommand{\Z}{\integers}
\newcommand{\rationals}{\mathbb{Q}}
\newcommand{\Q}{\rationals}
\newcommand{\reals}{\mathbb{R}}
\newcommand{\R}{\reals}
\newcommand{\complex}{\mathbb{C}}
\newcommand{\C}{\complex}
\newcommand{\halfPlane}{\mathbb{H}}
\let\H\relax
\newcommand{\H}{\halfPlane}
\newcommand{\comp}{^{\complement}}
\DeclareMathOperator{\Hom}{Hom}
\newcommand{\Ind}{\mathbbm{1}}
\newcommand{\cut}{\setminus}
\DeclareMathOperator{\elem}{elem}

% Graph Theory
\let\deg\relax
\DeclareMathOperator{\deg}{deg}
\newcommand{\degp}{\ptxt{deg}^{+}}
\newcommand{\degn}{\ptxt{deg}^{-}}
\newcommand{\precdot}{\mathrel{\ooalign{$\prec$\cr\hidewidth\hbox{$\cdot\mkern0.5mu$}\cr}}}
\newcommand{\succdot}{\mathrel{\ooalign{$\cdot\mkern0.5mu$\cr\hidewidth\hbox{$\succ$}\cr\phantom{$\succ$}}}}
\DeclareMathOperator{\cl}{cl}
\DeclareMathOperator{\affdim}{affdim}

% Probability
\newcommand{\Prob}{\mathbb{P}}
\newcommand{\Avg}{\mathbb{E}}
\DeclareMathOperator{\Var}{Var}
\DeclareMathOperator{\cov}{cov}

% Standard Math
\newcommand{\inv}{^{-1}}
\newcommand{\abs}[1]{\left|#1\right|}
\newcommand{\ceil}[1]{\left\lceil{}#1\right\rceil{}}
\newcommand{\floor}[1]{\left\lfloor{}#1\right\rfloor{}}
\newcommand{\conj}[1]{\overline{#1}}
\newcommand{\of}{\circ}
\newcommand{\tri}{\triangle}
\newcommand{\inj}{\hookrightarrow}
\newcommand{\surj}{\twoheadrightarrow}
\newcommand{\ndiv}{\nmid}
\renewcommand{\epsilon}{\varepsilon}
\newcommand{\divides}{\mid}
\newcommand{\ndivides}{\nmid}
\DeclareMathOperator{\lcm}{lcm}
\DeclareMathOperator{\sgn}{sgn}
\newcommand{\map}[4]{\!\!\!\begin{array}[t]{rcl}#1 & \!\!\!\!\to & \!\!\!\!#2\\ #3 & \!\!\!\!\mapsto & \!\!\!\!#4\end{array}}
\newcommand{\bigsum}[2]{\smashoperator[lr]{\sum_{\scalebox{#1}{$#2$}}}}

% Linear Algebra
\newcommand{\Id}{\textrm{\textnormal{Id}}}
\newcommand{\im}{\textrm{\textnormal{im}}}
\newcommand{\norm}[1]{\abs{\abs{#1}}}
\newcommand{\tpose}{^{T}}
\newcommand{\iprod}[1]{\left<#1\right>}
\DeclareMathOperator{\trace}{tr}
\newcommand{\chgBasMat}[3]{\!\!\tensor*[_{#1}]{\left[#2\right]}{_{#3}}}
\newcommand{\vecBas}[2]{\tensor*[]{\left[#1\right]}{_{#2}}}
\DeclareMathOperator{\GL}{GL}
\DeclareMathOperator{\Mat}{Mat}
\DeclareMathOperator{\vspan}{span}
\DeclareMathOperator{\rank}{rank}
\newcommand{\V}[1]{\vec{#1}}
\DeclareMathOperator{\proj}{proj}
\DeclareMathOperator{\compProj}{comp}
\DeclareMathOperator{\row}{row}

% Multilinear Algebra
\newcommand{\Lsym}{\L}
\let\L\relax
\DeclareMathOperator{\L}{\mathscr{L}}
\DeclareMathOperator{\A}{\mathcal{A}}
\DeclareMathOperator{\Alt}{Alt}
\DeclareMathOperator{\Sym}{Sym}
\newcommand{\ot}{\otimes}
\newcommand{\ox}{\otimes}
\DeclareMathOperator{\asc}{asc}
\DeclareMathOperator{\asSet}{set}
\DeclareMathOperator{\sort}{sort}
\DeclareMathOperator{\ringA}{\mathring{A}}

% Topology
\newcommand{\closure}[1]{\overline{#1}}
\newcommand{\uball}{\mathcal{U}}
\DeclareMathOperator{\Int}{Int}
\DeclareMathOperator{\Ext}{Ext}
\DeclareMathOperator{\Bd}{Bd}
\DeclareMathOperator{\rInt}{rInt}
\DeclareMathOperator{\ch}{ch}
\DeclareMathOperator{\ah}{ah}
\newcommand{\Tau}{\mathlarger{\mathlarger{\mathlarger{\mathlarger{\tau}}}}}

% Analysis
\DeclareMathOperator{\Graph}{Graph}
\DeclareMathOperator{\epi}{epi}
\DeclareMathOperator{\hypo}{hypo}
\DeclareMathOperator{\supp}{supp}
\newcommand{\lint}[2]{\underset{#1}{\overset{#2}{{\color{black}\underline{{\color{white}\overline{{\color{black}\int}}\color{black}}}}}}}
\newcommand{\uint}[2]{\underset{#1}{\overset{#2}{{\color{white}\underline{{\color{black}\overline{{\color{black}\int}}\color{black}}}}}}}
\newcommand{\alignint}[2]{\underset{#1}{\overset{#2}{{\color{white}\underline{{\color{white}\overline{{\color{black}\int}}\color{black}}}}}}}
\newcommand{\extint}{\ptxt{ext}\int}
\newcommand{\extalignint}[2]{\ptxt{ext}\alignint{#1}{#2}}
\newcommand{\conv}{\ast}

% Proofs
\newcommand{\st}{s.t.}
\newcommand{\unique}{!}
\newcommand{\iffdef}{\overset{\ptxt{def}}{\Leftrightarrow}}
\newcommand{\eqdef}{\overset{\ptxt{def}}{=}}
\newcommand{\eqVertical}{\rotatebox[origin=c]{90}{=}}
\newcommand{\mapsfrom}{\mathrel{\reflectbox{\ensuremath{\mapsto}}}}
\newcommand{\mapsdown}{\rotatebox[origin=c]{-90}{$\mapsto$}\mkern2mu}
\newcommand{\mapsup}{\rotatebox[origin=c]{90}{$\mapsto$}\mkern2mu}
\newcommand{\from}{\!\mathrel{\reflectbox{\ensuremath{\to}}}}

% Brackets
\newcommand{\paren}[1]{\left(#1\right)}
\renewcommand{\brack}[1]{\left[#1\right]}
\renewcommand{\brace}[1]{\left\{#1\right\}}
\newcommand{\ang}[1]{\left<#1\right>}

% Algorithms
\algrenewcommand{\algorithmiccomment}[1]{\hskip 1em \texttt{// #1}}
\algrenewcommand\algorithmicrequire{\textbf{Input:}}
\algrenewcommand\algorithmicensure{\textbf{Output:}}
\newcommand{\parSymbol}{\P}
\renewcommand{\P}{\ptxt{\textbf{P}}}
\newcommand{\NP}{\ptxt{\textbf{NP}}}
\newcommand{\NPC}{\ptxt{\textbf{NP-Complete}}}
\newcommand{\NPH}{\ptxt{\textbf{NP-Hard}}}
\newcommand{\EXP}{\ptxt{\textbf{EXP}}}

%%%%%%%%%%%%%%%%%%%%%%%%%%%%%
% Other commands
%%%%%%%%%%%%%%%%%%%%%%%%%%%%%
\newcommand{\flag}[1]{\textbf{\textcolor{red}{#1}}}
\newcommand{\uSym}{\u}
\let\u\relax
\newcommand{\u}[1]{\underline{#1}}
\newcommand{\bSym}{\b}
\let\b\relax
\newcommand{\b}[1]{\textbf{#1}}
\newcommand{\iSym}{\i}
\let\i\relax
\newcommand{\i}[1]{\textit{#1}}

%%%%%%%%%%%%%%%%%%%%%%%%%%%%%
% Make l's curvy in math environments
%%%%%%%%%%%%%%%%%%%%%%%%%%%%%
\mathcode`l="8000
\begingroup
\makeatletter
\lccode`\~=`\l
\DeclareMathSymbol{\lsb@l}{\mathalpha}{letters}{`l}
\lowercase{\gdef~{\ifnum\the\mathgroup=\m@ne \ell \else \lsb@l \fi}}%
\endgroup

\newcommand{\B}{
    \begin{tikzpicture}
    \filldraw [fill=red, draw=black] (0, 0) rectangle (0.37, 0.45);
    \draw [line width=0.5mm, white ] (0.1,0.08) -- (0.1,0.38);
    \draw[line width=0.5mm, white ] (0.1, 0.35) .. controls (0.2, 0.35) and (0.4, 0.2625) .. (0.1, 0.225);
    \draw[line width=0.5mm, white ] (0.1, 0.225) .. controls (0.2, 0.225) and (0.4, 0.1625) .. (0.1, 0.1);
    \end{tikzpicture}
}

\author{Professor David Barrett\\ \small\textit{Transcribed by Thomas Cohn}}
\title{Stokes' Theorem}
\date{2/11/19} % Can also use \today

\begin{document}
\maketitle
\setlength\RaggedRightParindent{\parindent}
\RaggedRight

\par\noindent If $\omega$ is a $C^{1}$ $(k-1)$-form on a neighborhood of a compact oriented manifold $M$, then $\int_{M}d\omega=\int_{\partial{}M}\omega$ with the induced orientation on $\partial{}M$.\n

\par\noindent Proof: Focus on the special case where $\supp\omega\subset{}V\overset{\alpha}{\from}U$ for orientation-preserving coordinate patch $\alpha$, we get the general case with finite sums. For
\[
\widetilde{\alpha^{*}\omega}\eqdef\left\{\begin{array}{ll}\alpha^{*}\omega & \quad\ptxt{on }U\\ 0 & \quad\ptxt{on }\H\cut{}U\end{array}\right.
\]
\[
\int_{M}d\omega=\int_{U}\alpha^{*}d\omega=\int_{U}d(\alpha^{*}\omega)=\int_{\H^{k}}d(\widetilde{\alpha^{*}\omega})
\]

\par\noindent Also note: $\displaystyle\int_{\partial{}M}\omega=\int_{U\cap\partial\H^{k}}\widetilde{\alpha^{*}\omega}$, so we can write\n
\[
\widetilde{\alpha^{*}\omega}=f_{1}\,dx_{2}\wedge\cdots\wedge{}dx_{k}+f_{2}\,dx_{1}\wedge{}dx_{3}\wedge\cdots\wedge{}dx_{k}+\cdots+f_{k}\,dx_{1}\wedge\cdots\wedge{}dx_{k-1}
\]

\par\noindent Thus,
\[
d(\widetilde{\alpha^{*}\omega})=(D_{1}f_{1}-D_{2}f_{2}+\cdots+(-1)^{k-1}D_{k}f_{k})dx_{1}\wedge\cdots\wedge{}dx_{k}
\]

\par\noindent So $\int_{\H^{k}}d(\widetilde{\alpha^{*}\omega})=\int_{\H^{k}}D_{1}f_{1}-D_{2}f_{2}+\cdots+(-1)^{k-1}D_{k}f_{k}$.\n
Replacing $\H^{k}$ with the box defined by $a_{1}\le{}x_{1}\le{}b_{1},\ldots,a_{k}\le{}x_{k}\le{}b_{k}$ (where $a_{k}=0$), such that every corner of the box is outside of $U$, yields
\begin{align*}
	\int_{\H^{k}}d(\widetilde{\alpha^{*}\omega}) & =\int_{a_{1}}^{b_{1}}\cdots\int_{a_{k}}^{b_{k}}D_{1}f_{1}-D_{2}f_{2}+\cdots+(-1)^{k-1}D_{k}f_{k}\\
	\ptxt{(Fubini)} & =\int_{a_{2}}^{b_{2}}\cdots\int_{a_{k}}^{b_{k}}\int_{a_{1}}^{b_{1}}D_{1}f_{1}-\int_{a_{1}}^{b_{1}}\cdots\int_{a_{k}}^{b_{k}}\int_{a_{2}}^{b_{2}}D_{2}f_{2}+\cdots+\int_{a_{1}}^{b_{1}}\cdots\int_{a_{k}}^{b_{k}}\int_{a_{k-1}}^{\smash{b_{k-1}}}D_{k-1}f_{k-1}+\int_{\H^{k}}D_{k}f_{k}\\
	\ptxt{(FTC)} & =\int_{a_{2}}^{b_{2}}\cdots\int_{a_{k}}^{b_{k}}(f_{1}(b_{1})-f_{1}(a_{1}))-\int_{a_{1}}^{b_{1}}\cdots\int_{a_{k}}^{b_{k}}(f_{2}(b_{2})-f_{2}(a_{2}))+\cdots+\int_{a_{1}}^{b_{1}}\cdots\int_{a_{k}}^{b_{k}}(f_{k-1}(b_{k-1})-f_{k-1}(a_{k-1}))+\int_{\H^{k}}D_{k}f_{k}\\
	& =0-0+\cdots+0+\int_{\H^{k}}D_{k}f_{k}\\
	\ptxt{(FTC)} & =(-1)^{k-1}(-1)\int_{\R^{k-1}}f_{k}\,dx_{1}\wedge\cdots\wedge{}dx_{k-1}\\
	& =\cancel{(-1)^{k}(-1)^{k}}\int_{\partial\H^{k}}f_{k}\,dx_{1}\wedge\cdots\wedge{}dx_{k-1}\qquad\qquad\ptxt{($(-1)^{k}$ comes from the induced orientation)}\\
	& =\int_{\partial\H^{k}}\alpha^{*}\omega\;\;\square\\
\end{align*}

\par\noindent\b{Revisiting Examples from Wednesday:}
\ex{Sphere $S^{2}$\n
$\displaystyle\int_{S^{2}}dx\wedge{}dy=\int_{\partial{}S^{2}}x\,dy=0$\n
$\displaystyle\int_{S^{2}}x\,dx\wedge{}dy=\int_{\partial{}S^{2}}\frac{x^{2}}{2}\,dy=0$\n
$\displaystyle\int_{S^{2}}z\,dx\wedge{}dy=\int_{B^{2}(1)}d(z\,dx\wedge{}dy)=\int_{B^{2}(1)}dz\wedge{}dx\wedge{}dy=\frac{4\pi}{3}$}

\ex{$\displaystyle{}M=\set{(x,y,z)\in\R^{3}\;:\;x^{2}+y^{2}=1,0\le{}z\le{}1}$, $\omega=\frac{-y\,dx+x\,dy}{x^{2}+y^{2}}=``d\theta"$ (so $d\omega=0$).\n
Then $\displaystyle\int_{M}d\omega=\int_{\partial{}M}\omega=2\pi-2\pi=0$, which is as expected, since $\displaystyle\int_{M}0=0$.}

\par\noindent\b{Long List of Orientation Special Cases}
\begin{enumerate}[label=(\arabic*)]
	\item $M^{n\ptxt{-mfd}}\subseteq\R^{n}$. Use standard orientation of $\R^{n}$ to get standard orientation of $M$.
	\item $X$ oriented $(n-1)$-mfd in $\R^{n}$ (perhaps $X=\partial{}M$ for some $M$ $n$-mfd).\n
	Related question: sorting out ``inside'' vs ``outside''.\n
	Recall: $\Tau_{\vec{p}}X$ is the column space of $D\alpha(\vec{q})$ (for $\vec{q}=\alpha\inv(\vec{p})$). $\dim\Tau_{\vec{p}}X=n-1$.
\end{enumerate}

\defn{$N_{\vec{p}}X\eqdef(\Tau_{\vec{p}}X)^{\perp}$ is called the \u{normal space}, and has dimension $1$.}
\begin{enumerate}[label=(\arabic*), start=2, topsep=0pt]
	\item (continued) Pick $\vec{N}(\vec{p})\in{}N_{\vec{p}}X$ \st{} $\norm{\vec{N}(\vec{p})}=1$ and $\det(\vec{N}(\vec{p})|D\alpha(\vec{p}))>0$ ($\star$).\n
	Convince yourself that this is independent of choice of orientation-preserving coordinate patch $\alpha$, and that $X\subset{}C^{r}\Rightarrow\vec{N}(\vec{p})\in{}C^{r-1}(X,\R^{n})$. (For the latter, see p315.)\n
	Conversely, given a coice of unit normal vector field for $X$, use ($\star$) as a criterion for $\alpha$ to be orientation-preserving, and get an orientation for $X$.
	\item[(2a)] $X=\partial{}M^{n\ptxt{-mfd}}$. Special case: $M=\H^{n}$. Then $\vec{N}(\vec{p})=-\vec{e_{n}}$ (exercise).\n
	General case: $\vec{N}(\vec{p})$ points out of $M$, i.e., $\vec{p}+\vec{N}(\vec{p})\not\in{}M$ for $0<t<\epsilon$.
\end{enumerate}

\end{document}