\documentclass[10pt,letterpaper]{article}
\usepackage[utf8]{inputenc}
\usepackage[intlimits]{amsmath}
\usepackage{amsfonts}
\usepackage{amssymb}
\usepackage{ragged2e}
\usepackage[letterpaper, margin=1in]{geometry}
\usepackage{graphicx}
\usepackage{cancel}
\usepackage{mathtools}
\usepackage{tabularx}
\usepackage{arydshln}
\usepackage{tensor}
\usepackage{array}
\usepackage{xcolor}
\usepackage[boxed]{algorithm}
\usepackage[noend]{algpseudocode}
\usepackage{listings}
\usepackage{textcomp}
\usepackage[pdf,tmpdir,singlefile]{graphviz}
\usepackage{mathrsfs}
\usepackage{bbm}
\usepackage{tikz}
\usepackage{enumitem}
\usepackage{arydshln}
\usepackage{relsize}

%%%%%%%%%%%%%%%%%%%%%%%%%%%%%
% Formatting commands
%%%%%%%%%%%%%%%%%%%%%%%%%%%%%
\newcommand{\n}{\hfill\break}
\newcommand{\up}{\vspace{-\baselineskip}}
\newcommand{\lemma}[1]{\par\noindent\settowidth{\hangindent}{\textbf{Lemma: }}\textbf{Lemma: }#1}
\newcommand{\defn}[1]{\par\noindent\settowidth{\hangindent}{\textbf{Defn: }}\textbf{Defn: }#1\n}
\newcommand{\thm}[1]{\par\noindent\settowidth{\hangindent}{\textbf{Thm: }}\textbf{Thm: }#1\n}
\newcommand{\prop}[1]{\par\noindent\settowidth{\hangindent}{\textbf{Prop: }}\textbf{Prop: }#1\n}
\newcommand{\cor}[1]{\par\noindent\settowidth{\hangindent}{\textbf{Cor: }}\textbf{Cor: }#1\n}
\newcommand{\ex}[1]{\par\noindent\settowidth{\hangindent}{\textbf{Ex: }}\textbf{Ex: }#1\n}
\newcommand{\exer}[1]{\par\noindent\settowidth{\hangindent}{\textbf{Exer: }}\textbf{Exer: }#1\n}
\newcommand{\proven}{\;$\square$\n}
\newcommand{\problem}[1]{\par\noindent{#1}\n}
\newcommand{\problempart}[2]{\par\noindent\indent{}\settowidth{\hangindent}{\textbf{(#1)} \indent{}}\textbf{(#1)} #2\n}
\newcommand{\ptxt}[1]{\textrm{\textnormal{#1}}}
\newcommand{\inlineeq}[1]{\centerline{$\displaystyle #1$}}
\newcommand{\pageline}{\noindent\rule{\textwidth}{0.1pt}}

%%%%%%%%%%%%%%%%%%%%%%%%%%%%%
% Math commands
%%%%%%%%%%%%%%%%%%%%%%%%%%%%%
% Set Theory
\newcommand{\card}[1]{\left|#1\right|}
\newcommand{\set}[1]{\left\{#1\right\}}
\newcommand{\setmid}{\;\middle|\;}
\newcommand{\ps}[1]{\mathcal{P}\left(#1\right)}
\newcommand{\pfinite}[1]{\mathcal{P}^{\ptxt{finite}}\left(#1\right)}
\newcommand{\naturals}{\mathbb{N}}
\newcommand{\N}{\naturals}
\newcommand{\integers}{\mathbb{Z}}
\newcommand{\Z}{\integers}
\newcommand{\rationals}{\mathbb{Q}}
\newcommand{\Q}{\rationals}
\newcommand{\reals}{\mathbb{R}}
\newcommand{\R}{\reals}
\newcommand{\complex}{\mathbb{C}}
\newcommand{\C}{\complex}
\newcommand{\halfPlane}{\mathbb{H}}
\let\H\relax
\newcommand{\H}{\halfPlane}
\newcommand{\comp}{^{\complement}}
\DeclareMathOperator{\Hom}{Hom}
\newcommand{\Ind}{\mathbbm{1}}
\newcommand{\cut}{\setminus}

% Graph Theory
\let\deg\relax
\DeclareMathOperator{\deg}{deg}
\newcommand{\degp}{\ptxt{deg}^{+}}
\newcommand{\degn}{\ptxt{deg}^{-}}
\newcommand{\precdot}{\mathrel{\ooalign{$\prec$\cr\hidewidth\hbox{$\cdot\mkern0.5mu$}\cr}}}
\newcommand{\succdot}{\mathrel{\ooalign{$\cdot\mkern0.5mu$\cr\hidewidth\hbox{$\succ$}\cr\phantom{$\succ$}}}}
\DeclareMathOperator{\cl}{cl}
\DeclareMathOperator{\affdim}{affdim}

% Probability
\newcommand{\Prob}{\mathbb{P}}
\newcommand{\Avg}{\mathbb{E}}

% Standard Math
\newcommand{\inv}{^{-1}}
\newcommand{\abs}[1]{\left|#1\right|}
\newcommand{\ceil}[1]{\left\lceil{}#1\right\rceil{}}
\newcommand{\floor}[1]{\left\lfloor{}#1\right\rfloor{}}
\newcommand{\conj}[1]{\overline{#1}}
\newcommand{\of}{\circ}
\newcommand{\tri}{\triangle}
\newcommand{\inj}{\hookrightarrow}
\newcommand{\surj}{\twoheadrightarrow}
\newcommand{\ndiv}{\nmid}
\renewcommand{\epsilon}{\varepsilon}
\newcommand{\divides}{\mid}
\newcommand{\ndivides}{\nmid}
\DeclareMathOperator{\lcm}{lcm}
\DeclareMathOperator{\sgn}{sgn}

% Linear Algebra
\newcommand{\Id}{\textrm{\textnormal{Id}}}
\newcommand{\im}{\textrm{\textnormal{im}}}
\newcommand{\norm}[1]{\abs{\abs{#1}}}
\newcommand{\tpose}{^{T}}
\newcommand{\iprod}[1]{\left<#1\right>}
\DeclareMathOperator{\trace}{tr}
\newcommand{\chgBasMat}[3]{\!\!\tensor*[_{#1}]{\left[#2\right]}{_{#3}}}
\newcommand{\vecBas}[2]{\tensor*[]{\left[#1\right]}{_{#2}}}
\DeclareMathOperator{\GL}{GL}
\DeclareMathOperator{\Mat}{Mat}
\DeclareMathOperator{\vspan}{span}
\DeclareMathOperator{\rank}{rank}
\newcommand{\V}[1]{\vec{#1}}
\DeclareMathOperator{\proj}{proj}
\DeclareMathOperator{\compProj}{comp}

% Multilinear Algebra
\newcommand{\Lsym}{\L}
\let\L\relax
\DeclareMathOperator{\L}{\mathscr{L}}
\DeclareMathOperator{\A}{\mathcal{A}}
\DeclareMathOperator{\Alt}{Alt}
\DeclareMathOperator{\Sym}{Sym}
\newcommand{\ot}{\otimes}

% Topology
\newcommand{\closure}[1]{\overline{#1}}
\newcommand{\uball}{\mathcal{U}}
\DeclareMathOperator{\Int}{Int}
\DeclareMathOperator{\Ext}{Ext}
\DeclareMathOperator{\Bd}{Bd}
\DeclareMathOperator{\rInt}{rInt}
\DeclareMathOperator{\ch}{ch}
\DeclareMathOperator{\ah}{ah}
\newcommand{\Tau}{\mathlarger{\mathlarger{\mathlarger{\mathlarger{\tau}}}}}

% Analysis
\DeclareMathOperator{\Graph}{Graph}
\DeclareMathOperator{\epi}{epi}
\DeclareMathOperator{\hypo}{hypo}
\DeclareMathOperator{\supp}{supp}
\newcommand{\lint}[2]{\underset{#1}{\overset{#2}{{\color{black}\underline{{\color{white}\overline{{\color{black}\int}}\color{black}}}}}}}
\newcommand{\uint}[2]{\underset{#1}{\overset{#2}{{\color{white}\underline{{\color{black}\overline{{\color{black}\int}}\color{black}}}}}}}
\newcommand{\alignint}[2]{\underset{#1}{\overset{#2}{{\color{white}\underline{{\color{white}\overline{{\color{black}\int}}\color{black}}}}}}}
\newcommand{\extint}{\ptxt{ext}\int}
\newcommand{\extalignint}[2]{\ptxt{ext}\alignint{#1}{#2}}
\newcommand{\conv}{\ast}

% Proofs
\newcommand{\st}{s.t.}
\newcommand{\unique}{!}
\newcommand{\iffdef}{\overset{\ptxt{def}}{\Leftrightarrow}}
\newcommand{\eqdef}{\overset{\ptxt{def}}{=}}
\newcommand{\eqVertical}{\rotatebox[origin=c]{90}{=}}
\newcommand{\mapsfrom}{\mathrel{\reflectbox{\ensuremath{\mapsto}}}}
\newcommand{\mapsdown}{\rotatebox[origin=c]{-90}{$\mapsto$}\mkern2mu}
\newcommand{\mapsup}{\rotatebox[origin=c]{90}{$\mapsto$}\mkern2mu}

% Brackets
\newcommand{\paren}[1]{\left(#1\right)}
\renewcommand{\brack}[1]{\left[#1\right]}
\renewcommand{\brace}[1]{\left\{#1\right\}}
\newcommand{\ang}[1]{\left<#1\right>}

% Algorithms
\algrenewcommand{\algorithmiccomment}[1]{\hskip 1em \texttt{// #1}}
\algrenewcommand\algorithmicrequire{\textbf{Input:}}
\algrenewcommand\algorithmicensure{\textbf{Output:}}
\newcommand{\parSymbol}{\P}
\renewcommand{\P}{\ptxt{\textbf{P}}}
\newcommand{\NP}{\ptxt{\textbf{NP}}}
\newcommand{\NPC}{\ptxt{\textbf{NP-Complete}}}
\newcommand{\NPH}{\ptxt{\textbf{NP-Hard}}}
\newcommand{\EXP}{\ptxt{\textbf{EXP}}}

%%%%%%%%%%%%%%%%%%%%%%%%%%%%%
% Other commands
%%%%%%%%%%%%%%%%%%%%%%%%%%%%%
\newcommand{\flag}[1]{\textbf{\textcolor{red}{#1}}}

%%%%%%%%%%%%%%%%%%%%%%%%%%%%%
% Make l's curvy in math environments
%%%%%%%%%%%%%%%%%%%%%%%%%%%%%
\mathcode`l="8000
\begingroup
\makeatletter
\lccode`\~=`\l
\DeclareMathSymbol{\lsb@l}{\mathalpha}{letters}{`l}
\lowercase{\gdef~{\ifnum\the\mathgroup=\m@ne \ell \else \lsb@l \fi}}%
\endgroup

\newcommand{\B}{
    \begin{tikzpicture}
    \filldraw [fill=red, draw=black] (0, 0) rectangle (0.37, 0.45);
    \draw [line width=0.5mm, white ] (0.1,0.08) -- (0.1,0.38);
    \draw[line width=0.5mm, white ] (0.1, 0.35) .. controls (0.2, 0.35) and (0.4, 0.2625) .. (0.1, 0.225);
    \draw[line width=0.5mm, white ] (0.1, 0.225) .. controls (0.2, 0.225) and (0.4, 0.1625) .. (0.1, 0.1);
    \end{tikzpicture}
}

\author{Professor David Barrett\\ \small\textit{Transcribed by Thomas Cohn}}
\title{Multilinear Algebra}
\date{1/23/19} % Can also use \today

\begin{document}
\maketitle
\setlength\RaggedRightParindent{\parindent}
\RaggedRight

\par\noindent Recall: a $k$-tensor on vector space $V$ is a multilinear map $f:\underbrace{V\times\cdots\times{}V}_{V^{k}}\to\R$.

\defn{$\L^{k}(V)$ is defined to be the set of all $k$-tensors on $V$.}

\defn{$\Sym^{k}(V)\eqdef\set{f\in\L^{k}(V):f\ptxt{ is symmetric}}$}

\defn{$\A^{k}(V)\eqdef\set{f\in\L^{k}(V):f\ptxt{ is alternating}}$. Sometimes written as $\Alt^{k}(V)$.}

\par\noindent Recall: $\L^{1}(V)=\Sym^{1}(V)=A^{1}=V^{*}$.\n

\par\noindent Suppose $\vec{a_{1}},\ldots,\vec{a_{n}}$ are a basis for $V$. We can write $\vec{a}\in{}V$ as $\vec{a}=\sum_{j=1}^{n}c_{j}\vec{a_{j}}$. So for $f\in\L^{k}$,
\begin{align*}
	f(\vec{v_{1}},\ldots,\vec{v_{k}}) & =f\paren{\sum_{j_{1}=1}^{n}c_{1,j_{1}}\vec{a_{j_{1}}},\ldots,\sum_{j_{k}=1}^{n}c_{k,j_{k}}\vec{a_{j_{k}}}}\\
	& =c_{1,1}f\paren{\vec{a_{1}},\sum_{j_{2}=1}^{n}c_{2,j_{2}}\vec{a_{j_{2}}},\ldots,\sum_{j_{k}=1}^{n}c_{n,j_{k}}\vec{a_{j_{k}}}}\\
	& \phantom{=}+c_{1,2}f\paren{\vec{a_{2}},\sum_{j_{2}=1}^{n}c_{2,j_{2}}\vec{a_{j_{2}}},\ldots,\sum_{j_{k}=1}^{n}c_{n,j_{k}}\vec{a_{j_{k}}}}\\
	& \phantom{=}+\quad\vdots\\
	& \phantom{=}+c_{1,n}f\paren{\vec{a_{n}},\sum_{j_{2}=1}^{n}c_{2,j_{2}}\vec{a_{j_{2}}},\ldots,\sum_{j_{k}=1}^{n}c_{n,j_{k}}\vec{a_{j_{k}}}}\\
	& \cdots=\sum_{j_{1},j_{2},\ldots,j_{k}\in\set{1,\ldots,n}}c_{1,j_{1}}c_{2,j_{2}}\cdots{}c_{k,j_{k}}f(\vec{a_{j_{1}}},\ldots,\vec{a_{j_{k}}})
\end{align*}

\par\noindent Where the final sum is over every possible combination of one coefficient from every $j_{l}$.\n
So $f$ is determined by $f(\vec{a_{j_{1}}},\ldots,\vec{a_{j_{k}}})$.\n

\par\noindent Let $I=(i_{1},\ldots,i_{k})\in\set{1,\ldots,n}^{k}=\underbrace{\set{1,\ldots,n}\times\cdots\times\set{1,\ldots,n}}_{k}$.

\defn{$\displaystyle\phi_{I}:\paren{\sum_{j_{1}=1}^{n}c_{1,j_{1}}\vec{a_{j_{1}}},\ldots,\sum_{j_{k}=1}^{n}C_{k,j_{k}}\vec{a_{j_{k}}}}\mapsto{}c_{1,j_{1}}\cdots{}c_{k,j_{k}}$.}

\exer{$\phi_{I}\in\L^{k}(V)$}

\exer{$\phi_{I}(\vec{a_{J}})=\left\{\begin{array}{ll}1 & \quad{}I=J\\ 0 & \quad{}I\ne{}J\end{array}\right.$}

\exer{$\set{\phi_{I}:I\in\set{1,\ldots,n}^{k}}$ is linearly independent.}

\prop{$\displaystyle{}f\in\L^{k}(V)\Rightarrow{}f=\sum_{I}f(\vec{a_{I}})\phi_{I}$\n
Proof follows from the uniqueness result before.}

\par\noindent Thus, $\set{\phi_{I}}_{I\in\set{1,\ldots,n}^{k}}$ is a \underline{basis} for $\L^{k}(V)$. $\dim(\L^{k}(V))=n^{k}=(\dim{}V)^{k}$.\n

\par\noindent We can pick any constants $C_{I}$, and get $f=\sum{}c_{I}\phi_{I}$ with $f(\vec{a_{I}})=c_{I}$.\n

\defn{Given $f\in\L^{k}(V),g\in\L^{l}(V)$, we say $f\ot{}g\in\L^{k+l}(V)$, where\n
$(f\ot{}g)(\vec{v_{1}},\ldots,\vec{v_{k+l}})=f(\vec{v_{1}},\ldots,\vec{v_{k}})g(\vec{v_{k+1}},\ldots,\vec{v_{k+l}})$.}

\par\noindent Some rules:
\begin{itemize}[label=-,noitemsep,partopsep=0pt,topsep=0pt,parsep=0pt]
	\item $f\ot(g\ot{}h)=(f\ot{}g)\ot{}h$
	\item $(cf)\ot{}g=c(f\ot{}g)=f\ot(cg)$
	\item $(f+g)\ot{}h=(f\ot{}h)+(g\ot{}h)$
	\item $f\ot(g+h)=(f\ot{}g)+(f\ot{}h)$
\end{itemize}

\par\noindent\n
Also, $\phi_{I}=\phi_{i_{1}}\ot\cdots\ot\phi_{i_{k}}$ with each $\phi_{i_{l}}\in\L^{1}(V)=V^{*}$.\n

\par\noindent So $T:V\to{}W$ linear induces $T^{*}:\L^{k}(W)\to\L^{k}(V)$ by $(T^{*}f)(\vec{v_{1}},\ldots,\vec{v_{k}})=f(T(\vec{v_{1}}),\ldots,T(\vec{v_{k}}))$. Rules:
\begin{itemize}[label=-,noitemsep,partopsep=0pt,topsep=0pt,parsep=0pt]
	\item $T^{*}$ is linear
	\item $T^{*}(f\ot{}g)=T^{*}f\ot{}T^{*}g$
	\item $(S\of{}T)^{*}f=T^{*}(S^{*}f)$
\end{itemize}

\par\noindent\n
Let $f\in\Alt^{k}(V)$ and $\card{\set{i_{1},\ldots,i_{k}}}<k$ (i.e., there's a repitition). Then $f(a_{I})=0$.\n

\defn{$I\in\set{1,\ldots,n}^{k}$ is \underline{ascending} $\iffdef$ $1\le{}i_{1}<i_{2}<\cdots<i_{k}$}

\defn{$J=\set{1,\ldots,n}^{k}$ and $\card{\set{j_{1},\ldots,j_{k}}}=k$ $\Rightarrow$ $\exists\unique$ ascending $\sigma\in{}S_{k}$ \st{} $J=I_{\sigma}\eqdef(i_{\sigma(1)},\ldots,i_{\sigma(k)})$.}

\begin{align*}
	f & =\sum_{I\in\set{1,\ldots,n}^{k}}f(a_{I})\phi_{I}\\
	& =\sum_{I\in\set{1,\ldots,n}^{k}\ptxt{ asc.}}\brack{\sum_{\sigma\in{}S_{k}}f(\vec{a_{I_{\sigma}}})\phi_{I_{\sigma}}}\\
	& =\sum_{I\ptxt{ asc.}}\brack{\sum_{\sigma\in{}S_{k}}\sgn_{\sigma}f(\vec{a_{I}})\phi_{I_{\sigma}}}\\
	& =\sum_{I\ptxt{ asc.}}\left[\vphantom{\sum_{\sigma\in{}S_{k}}}\right.f(\vec{a_{I}})\underbrace{\sum_{\sigma\in{}S_{k}}\sgn_{\sigma}\phi_{I_{\sigma}}}_{\eqdef\psi_{I}}\left.\vphantom{\sum_{\sigma\in{}S_{k}}}\right]\\
	& =\sum_{I\ptxt{ asc.}}f(\vec{a_{I}})\psi_{I}
\end{align*}

\par\noindent Thus, $\set{\psi_{I}}_{I\ptxt{ asc.}}$ forms a basis for $\A^{k}(V)$.\n
$k>n\Rightarrow\A^{k}(V)=\set{0}$.\n
$1\le{}k\le{}n\Rightarrow\dim\A^{k}(V)$ is the number of ascending $k$-tuples in $\set{1,\ldots,n}$.\n

\par\noindent Specialize to $V=\R^{n}$.\n
For $M=(\vec{x_{1}},\ldots,\vec{x_{n}})\in\Mat(n,n)$, we have $\det(M)=\psi_{(1,\ldots,n)}(\vec{x_{1}},\ldots,\vec{x_{n}})$.\n

\par\noindent Check: $\det{}I_{n}=\psi_{(1,\ldots,n)}(\vec{e_{1}},\ldots,\vec{e_{n}})=1$.\n

\par\noindent We want some sort of product $\A^{k}\wedge\A^{l}\to\A^{k+l}$.

\end{document}